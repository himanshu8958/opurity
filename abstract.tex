\begin{abstract}
Purity is an important concept in the context of an imperative
programming language, with multiple applications.  We present a
semantics-based definition of \emph{observational purity} for an
imperative procedure, formalizing the intuition that the procedure's
execution has only benign side effects, \emph{i.e.}, side-effects that
cannot be observed and
do not affect the input-output behaviour of the procedure.

We then address two mutually-dependent problems: verifying that a
procedure is observationally pure and verifying assertions/contracts
that contain calls to a presumed-to-be observationally pure procedure.
As we illustrate, verification that a procedure is observationally
pure can benefit from assertions that contain calls to the same
procedure.  However, such assertions themselves are meaningful only if
the procedure is observationally pure.
Thus, the two problems cannot be decoupled and must be addressed
simultaneously.


We present two techniques for performing this dual verification.
\end{abstract}