\begin{abstract}
  Verifying whether a procedure is \emph{observationally pure} is useful in
  many software engineering scenarios. An observationally pure procedure
  always returns the same value for the same argument, and thus mimics a
  mathematical function. The problem is challenging when procedures use
  private mutable global variables, e.g., for memoization of frequently
  returned answers, and when they involve recursion.

  We present a novel verification approach for this problem. Our approach
  involves encoding the procedure's code as a formula that is a disjunction
  of path constraints, with the recursive calls being replaced in the
  formula with references to a mathematical function symbol. Then, a
  theorem prover is invoked to check whether the formula that has been
  constructed agrees with the function symbol referred to above in terms of
  input-output behavior for all arguments.

  %% Key to making this idea work is a
  %% user-provided invariant, which is used as an ``assume'' 
  %% at the points just after the recursive calls, and which summarizes the
  %% possible values that could have been assigned to the global variables
  %% within the recursive calls. In order to make it easy to construct this
  %% invariant, we allow the invariant itself to also refer to the function
  %% symbol referred to above if necessary. 

  We evaluate our approach on a set of realistic examples, using the Boogie
  intermediate language and theorem prover. Our evaluation shows that the
  invariants are easy to construct manually, and that our approach is
  effective at verifying observationally pure procedures.
\end{abstract}
