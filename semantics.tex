\newcommand{\iosem}[1]{\rightsquigarrow_{#1}}
\newcommand{\sssem}[1]{\rightarrow_{#1}}

\section{Semantics}

The input-output semantics of a procedure $\proc$ is formalized as a relation $\iosem{\proc}$, where $n \iosem{\proc} r$ indicates
that an invocation of $\proc$ with input $n$ may return a result of $r$. The procedure is defined to be observationally
pure if the relation $\iosem{\proc}$ is a (partial) function: that is, if 
$n \iosem{\proc} r_1$ and $n \iosem{\proc} r_2$, then $r_1 = r_2$.

We introduce several conceptsas part of the semantic formalism.

\TODO{Formalize notion of a state}
A procedure $\proc$ determines a single-step execution relation $\sssem{\proc}$, where $\sigma_1 \sssem{\proc} \sigma_2$ indicates
that execution proceeds from state $\sigma_1$ to state $\sigma_2$ in a single step.
We define an \emph{execution} to be a sequence of states $\sigma_0 \sigma_1 \cdots \sigma_n$ such that
$\sigma_i \sssem{\proc} \sigma_{i+1}$ for all $0 \leq i < n$.

A \emph{trace} is a representation of a complete execution of the procedure produced by a single
invocation of the procedure. Note that a trace may contain nested sub-traces,
which are themselves traces, due to recursive calls.
A \emph{history} is a representation of the complete  execution of the procedure produced by a
sequence of invocations of the procedure.
(In this paper we consider only \emph{sequential} executions of the procedure. Thus, no concurrent invocations
of the procedure by a concurrent client of the library is allowed.)

\TODO{Older definitions appear below.}

\begin{definition}[at-entry(history/trace/trace-segment)  $\rightarrow$
    value] 
  $\\\atEntry{x}$ denotes the value of global variable $\g$ at the
  starting point of $x$ where $x$ is a history, trace or trace
  segment.
\end{definition}

\begin{definition}[trace]
  A trace $\trace$ is a complete execution of the procedure starting
  from a given global state and a tuple of formal arguments. A trace
  upon completion gives an end state and a return value.\\ Traces may
  have sub-traces, which by definition are traces (recursive calls).
\end{definition}

\begin{definition}[history]
  A history $\history$ is a sequence of consecutive traces, with the
  first trace starting from a given global state, and each subsequent
  trace begins in the state in which the previous trace ends.
\end{definition}

\begin{definition}[histories$\mhyphen$of (global state)$\rightarrow$
    set of histories] 
  $\historyOf{\gstate}$ returns the set of all possible histories such
  that for each history $\history$ in the set, $\history$ is a history
  of procedure `\foo' with $\atEntry{\history} = \gstate$.
\end{definition}

\begin{definition}[traces$\mhyphen$Of(history) $\rightarrow$ set of
    traces] $\tracesOf{\history}$ returns the set of traces and
  sub-traces in history $\history$.
\end{definition}

Invariant $\inv$ is assumed at the beginning of the procedure and
asserted at the end. Also, $\inv$ is asserted before the procedure
call and assumed after the procedure call. Above mentioned treatment
is similar to handling of pre-conditions and post-conditions for
recursive procedures. For the invariant, it is assumed that the given
procedure is observationally pure. The invariant has the free variable
$\F$ which is also used to abstract over procedure calls and it is
defined only if the given procedure is observationally pure.

Let $\gstate$ be a global state, and $\F'$ be a function, and $\inv$
be a invariant. If we substitute $\gstate$ for $g$ and $\F'$ for $\F$
in $\inv$ and it is true, then we say that $(\gstate, \F')$ satisfies
$\inv$, which is denoted as $(\gstate, \F') \satisfies \inv$. For
instance let $\inv := g = -1 \vee g = \F(18) * 3$ then $(3!, \lambda
n. n!) \satisfies \inv$.

\section{Observational purity}\label{sec:op}

\begin{definition}[argument(trace $\rightarrow$ value)]
  $\param{\trace}$ denotes the mapping of the argument `$\n$' to its
  value in the beginning of $\trace$.
\end{definition}


\begin{definition}[observational purity(set of states)]
  A procedure `\foo' is observational pure wrt a given set of global
  states $\Gstates$ if for each history $\history$ of procedure
  `\foo', all pairs of traces in the history that begin in a global
  states $\gstate$ and $\gstate'$ respect each other where $\gstate
  \in \Gstates$, $\gstate' \in \Gstates$.  In other words
  $\forall \gstate \in \Gstates.\forall \, \history \in
  \historyOf{\gstate}. \forall \trace_1 \in
  \tracesOf{\history}. \forall \trace_2 \in
  \tracesOf{\history}.\\ \param{\trace_1} = \param{\trace_2} \implies
  \return{\trace_1} = \return{\trace_2}$.
\end{definition}

\begin{lstlisting}[caption={Procedure `remember' : always returns the
      argument from its first call}, label=lst:remember]
int g = 0;
int init = 0;
int remember( int n) {
  if(init == 0){
    init = 1;
    g = n;
  }
  return g;
}
\end{lstlisting}

In a history, if all traces return the same result for an argument
value, there exists a function $\F$ that respects the given
procedure. In a history, the given procedure mimics a unique function,
but the mimicked function varies across histories. If the invariant is
a set of concrete states then there exists a unique function that is
equivalent to the given procedure across histories. But, our invariant
allows function symbols. Use of the function symbol allows an OP
procedure to mimic different functions across histories. For instance,
procedure `remember' in Listing~\ref{lst:remember} mimics a constant
function for $\inv := g=0 \wedge init=0 \vee g= \F() \wedge
init=1$. Whereas, for the invariant $\inv := g=0 \wedge init=0 \vee
g=2 \wedge init=1$; the procedure mimics the function $\lambda n.2$.
