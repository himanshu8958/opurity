\section{Proofs}

\begin{lemma}
If there exists no impurity witness $(\pi_1,\pi_2)$ and there exists
no $\inv$-violation witness $\pi$, then $\proc$ satisfies $\pureinv$.
\end{lemma}

\begin{proof}
Consider the set $\Pi$ of all feasible executions of $\proc$.
If there exists no impurity witness, then there exists a function $f$ compatible with $\Pi$.
(We take the partial function $\{ (\inputval{\pi}, \outputval{\pi}) \; | \; \pi \in \Pi \}$ and
extend it to be a total function.)
Further, since there exists no $\inv$-violation witness, we are guaranteed that $(\pi,f) \models \inv$
for every $\pi \in \Pi$.
\end{proof}

\begin{lemma}
If there exists a $\inv$-violation witness $\pi$, then there exists a minimal $\inv$-violation witness.
\end{lemma}

\begin{lemma}
If there exists an impurity witness, then there exists a minimal impurity witness or a minimal
$\inv$-violation witness.
\end{lemma}
