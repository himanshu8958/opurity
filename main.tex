% This is LLNCS.DEM the demonstration file of
% the LaTeX macro package from Springer-Verlag
% for Lecture Notes in Computer Science,
% version 2.4 for LaTeX2e as of 16. April 2010
%
\documentclass{llncs}
%
\usepackage{makeidx}  % allows for indexgeneration
%% himanshu
\usepackage[title]{appendix}

\usepackage{algorithm}
\usepackage{algorithmic}
\usepackage{amsmath}

\usepackage{listings}
\usepackage{syntax}
\usepackage{amssymb}
\usepackage{mathpartir}

\newcommand{\TODO}[1]{\textbf{TODO: #1}}
\newcommand{\eg}{\emph{e.g.}}

% \newcommand{\code}[1]{\ensuremath{\mathtt{#1}}}
\newcommand{\code}[1]{\texttt{#1}}

\newcommand{\var}{\code{x}}
\newcommand{\varY}{\code{y}}
\newcommand{\expr}{\code{e}}

\newcommand{\proc}{\textsc{p}}
\newcommand{\procP}{\textsc{p}}

\newcommand{\stmt}{\code{S}}
\newcommand{\stmtS}{\code{S}}
\newcommand{\stmtSA}{{\ensuremath{\code{S}_1}}}
\newcommand{\stmtSB}{{\ensuremath{\code{S}_2}}}
\newcommand{\stmtSC}{{\ensuremath{\code{S}_3}}}

\newcommand{\logicalformula}[1]{\varphi^{#1}}
\newcommand{\inv}{\logicalformula{inv}}
\newcommand{\post}{\logicalformula{post}}
\newcommand{\postA}{\post_1}
\newcommand{\postB}{\post_2}
\newcommand{\pre}{\logicalformula{pre}}
\newcommand{\vc}{\logicalformula{vc}}
\newcommand{\vcA}{\vc_1}
\newcommand{\vcB}{\vc_2}
\newcommand{\mainvcfn}{\textsc{postvc}}
\newcommand{\mainvc}[2]{\mainvcfn({#1},{#2})}
\newcommand{\auxvcfn}{\textsc{vc}}
\newcommand{\auxvc}[3]{\auxvcfn({#1},{#2},{#3})}

\newcommand{\foo}{\textit{`foo'}}
\newcommand{\trace}{\pi}
% \newcommand{\inv}{\mathit{inv}}
\newcommand{\history}{\Pi}
\newcommand{\pathCondition}{\mathit{T_{\foo}}}
\newcommand{\OPCheckE}{\mathit{OPCheck \mhyphen existential}}
\newcommand{\OPCheckA}{\mathit{OPCheck \mhyphen pairwise}}
\mathchardef \mhyphen="2D
\newcommand{\mi}[1]{\mathit{#1}}
\newcommand{\retVar}{\textit{retVar}}
\newcommand{\F}{\mathit{F}}
\newcommand{\n}{\textit{n}}
\newcommand{\gstate}{\gamma}
\newcommand{\g}{\textit{g}}
\newcommand{\gout}{\textit{gout}}
\newcommand{\gbef}{\textit{gbef}}
\newcommand{\gaft}{\textit{gaft}}
\newcommand{\satisfies}{\vdash}
\newcommand{\fact}{\mathit{fact}}
\newcommand{\formula}{\mu}
\newcommand{\integer}{\mathbb{N}}
\newcommand{\vci}[2]{\mathit{valid \mhyphen consistent \mhyphen
    invariant(#1, #2)}}
\newcommand{\atEntry}[1]{\mathit{at \mhyphen entry(#1)}}
\newcommand{\beforeCall}[1]{\mathit{before \mhyphen call(#1)}}
\newcommand{\return}[1]{\mathit{return(#1)}}
\newcommand{\atEnd}[1]{\mathit{at \mhyphen end(#1)}}
\newcommand{\tSegment}{\tau}
\newcommand{\param}[1]{\mathit{argument(#1)}}
\newcommand{\segDepth}[2]{\mathit{segment \mhyphen depth(#1, #2)}} %trace, segment
\newcommand{\subTrace}{\pi_{sub}}
\newcommand{\segment}[2]{\mathit{segment(#1, #2)}} %trace, N
\newcommand{\pairFormula}{\mathit{f_{pairwise}}}
\newcommand{\depth}[1]{\mathit{depth(#1)}}
\newcommand{\level}[1]{\mathit{level(#1)}}
\newcommand{\state}{\sigma}
\newcommand{\afterCall}[1]{\mathit{after \mhyphen call(#1)}} 

\newcommand{\ivars}{\overrightarrow{i}}
\newcommand{\ovars}{\overrightarrow{o}}
\newcommand{\avars}{\overrightarrow{a}}

\newcommand{\histories}[1]{\mathit{histories(#1)}}
\newcommand{\Gstates}{\Gamma}

\newcommand{\historyOf}[1]{\mathit{histories \mhyphen of(#1)}}
\newcommand{\tracesOf}[1]{\mathit{traces \mhyphen of(#1)}} 

\lstdefinestyle{mystyle}{
  basicstyle=\footnotesize, 
  %% breakatwhitespace=false,         
  %% breaklines=true,                 
  captionpos=b,
  %% keepspaces=true,         
  numbers=left,
  numbersep=6pt,                  
  %% showspaces=false,                
  %% showstringspaces=false,
  %% showtabs=false,
  morekeywords={if, for, else,
  procedure,modifies, var, returns, int, call, assume},
  %% tabsize=2
}
\lstset{
  %% basicstyle=\itshape,
  style = mystyle,
  xleftmargin=3em,
  %% literate={\alt}{;}{1}
  %% basicstyle=\footnotesize,
  numberstyle=\tiny,
  breaklines=true,
  escapeinside={\%*}{*)}
}

%
\begin{document}
%
\frontmatter          % for the preliminaries
%
\pagestyle{headings}  % switches on printing of running heads
% \addtocmark{Hamiltonian Mechanics} % additional mark in the TOC
%% \tableofcontents
%% %
\mainmatter              % start of the contributions
%
\title{Checking Observational Purity of Procedures}
%
\titlerunning{Checking Observational Purity}  % abbreviated title (for running head)
%                                     also used for the TOC unless
%                                     \toctitle is used
%
%% \author{Himanshu~Arora\inst{1}\orcidID{0000-1111-2222-3333} \and
%% Raghavan~Komondoor\inst{1}\orcidID{1111-2222-3333-4444} \and
%% G.~Ramalingam\inst{2}\orcidID{2222-3333-4444-5555}
%% }

\author{Himanshu Arora\inst{1} \and
Raghavan Komondoor\inst{1} \and
G. Ramalingam\inst{2}
}
%
% \authorrunning{Himanshu et al.}
%
 \institute{Indian Institute of Science, Bangalore\\
  \email{\{himanshua, raghavan\}@iisc.ac.in},
 \and
 Microsoft Research \\
 \email{grama@microsoft.com}}

\maketitle              % typeset the title of the contribution

\pagestyle{empty}
\begin{abstract}
  Verifying whether a procedure is \emph{observationally pure} 
  (that is, it always returns the same result for the same input argument)
  is challenging when the procedure uses mutable (private) global variables,
  e.g., for memoization, and when the procedure is recursive.

  We present a deductive verification approach for this problem. Our approach
  encodes the procedure's code as a logical formula, with recursive calls being
  modeled using a mathematical function symbol \emph{assuming that the
  procedure is observationally pure}. Then, a theorem prover is invoked to
  check whether this logical formula agrees with the
  function symbol referred to above in terms of input-output behavior for all arguments.
  We prove the soundness of this approach.

  We then present a conservative approximation of the first approach that reduces
  the verification problem to one of checking whether a quantifier-free formula
  is satisfiable and prove the soundness of the second approach.
  %% Key to making this idea work is a
  %% user-provided invariant, which is used as an ``assume'' 
  %% at the points just after the recursive calls, and which summarizes the
  %% possible values that could have been assigned to the global variables
  %% within the recursive calls. In order to make it easy to construct this
  %% invariant, we allow the invariant itself to also refer to the function
  %% symbol referred to above if necessary. 

  We evaluate our approach on a set of realistic examples, using the Boogie
  intermediate language and theorem prover. Our evaluation shows that the
  invariants are easy to construct manually, and that our approach is
  effective at verifying observationally pure procedures.
\end{abstract}

\section{Introduction}

A procedure in an imperative programming language is said to be
\emph{observationally pure} (OP) if for each specific argument value it has
a specific return value, across all possible sequences of calls to the
procedure, irrespective of what other code runs between these calls.  In
other words, the input-output behavior of an OP procedure mimics a mathematical
function.



% Observational purity is a useful property, as OP procedures can be memoized, and
% statements that call OP procedures can be subject to code motion, code
% optimization, etc.

Any procedure whose code is deterministic and does not read any
pre-existing state other than its arguments is trivially OP.
However, it is common for procedures, especially ones in libraries,
to update and read global variables, typically to optimize their own behavior,
while still mimicking mathematical functions in terms of their input-output behavior.
% At the same time, they also use looping or recursion.
In this paper, we focus on the problem of checking observational purity of
procedures that read and write global variables, especially in the presence of recursion,
which makes the problem harder.

\subsection{Motivating Example}
We use the example procedure `factCache' in
Listing~\ref{lst:factorialSimple}  as our running example. It
returns n! for the given argument n, and caches the return value for the
last argument provided to it. It uses two ``private'' global variables to
implement the caching -- \code{g}, and \code{lastN}. \code{g} is initialized to -1, and after
the first call to the procedure onwards is used to store the return value
from the most recent call. \code{lastN} is used to store the argument received in
the most recent call. Clearly this procedure is OP, and mimics the
input-output behavior of a regular factorial procedure that does not cache
any results. 

\begin{lstlisting}[float,language=c,basicstyle=\scriptsize,caption= {Procedure factCache:
      returns n!, and memoizes most recent result.},
    label=lst:factorialSimple]
  
int g := -1;
int lastN := 0;
int factCache( int n) {
  if(n <= 1) {
    result := 1;
  } else if (g != -1 && n == lastN) {
    result := g;
  } else {
    g = n * factCache( n - 1 );
    lastN = n;
    result := g;
  }
  return result;
}
\end{lstlisting}

\subsection{Proposed Approach}
\label{ssec:intro:approach}
%
%% 1. We are looking for a Hoare logic style approach, for increased
%%    precision. Herein, how to process the recursive calls? One solution is
%%    to make the user guess the mathematical function implemented by the
%%    procedure as well as the invariant on the global variables. Now, the
%%    recursive call can be replaced by these specifications, and the whole
%%    procedure can be checked whether it always returns n!.
%
%% 2. The approach above would itself be novel. However, one issue is that
%%    theorem provers face difficulty in proving equivalence of complex
%%    formulas. Also, humans face difficulty in inferring meanings of
%%    procedures. We therefore propose a more syntactic approach that is
%%    easier on the theorem prover and on the human. 
%
% Floyd-Hoare logic has been commonly used in previous approaches that try to
% tackle verification tasks with good precision.
Our verification approach is based on Floyd-Hoare logic. In order to verify
a recursive procedure inductively, typically a specification of the
procedure would need to be provided. The first idea for such a
specification would be a full functional specification of the
procedure. That is, if someone specifies that factCache mimics n!, then the
verifier could replace Line~10 in the code with `g = n * (n-1)!'. This, on
paper, is sufficient to assert that Line 12 always assigns n! to
`result'. However, to establish that Line~8 also does the same, an invariant
would need to be provided that describes the possible values of g before
any invocation to the procedure. In our example, a suitable invariant would
be `(g = -1) $\vee$ (g = lastN!)'. The verifier would also need to verify
that at the procedure's exit the invariant is re-established. Lines 10-12,
with the recursive call replaced by (n-1)!, suffices on paper to
re-establish the invariant.

The candidate approach mentioned above, while being plausible, suffers from
two weaknesses. The first is that a human would need to guess the
mathematical expression that is implemented by the given procedure. This may not be
easy when the procedure is complex. Second, the underlying theorem prover
would need to prove complex arithmetic properties, e.g., that n * (n-1)! is
equal to n!. Complex proofs such as this may be out of bounds for many
existing theorem provers.

Our key insight is to sidestep the challenges mentioned by introducing a
function symbol, say \emph{factCache}, and replacing the recursive call for the purposes
of verification with this function symbol. Intuitively, \emph{factCache} represents the
mathematical function that the given procedure mimics if the procedure is
OP.  In our example, Line~10 would become `g = n * \emph{factCache}(n-1)'. This step
needs no human involvement. The approach needs an invariant; however, in a
novel manner, we allow the invariant also to refer to \emph{factCache}. In our example,
a suitable invariant would be `(g = -1) $\vee$ (g = lastN *
\emph{factCache}(lastN-1))'. This sort of invariant is relatively easy to construct;
e.g., a human could arrive at it just by looking at Line~2 and with a local
reasoning on Lines~10 and~11. Given this invariant, (a) a theorem prover
could infer that the condition in Line~7 implies that Line~8 necessarily
copies the value of `n * \emph{factCache}(n-1)' into `result'. Due to the transformation to Line~10 mentioned above,
(b) the theorem prover can infer that Line~12 also does the same. Note that since these two expressions are syntactically
identical, a theorem prover can easily establish that they are equal in
value.  Finally, since Line~6 is reached under a different condition than
Lines~8 and~12, the verifier has finished establishing that the procedure
always returns the same expression in n for any given value of n.

Similarly, using the modified Line~10 mentioned above and from Line~11, the
prover can re-establish that g is equal to `lastN * \emph{factCache}(lastN - 1)' when
control reaches Line~12. Hence, the necessary step of proving the given
invariant to be a valid invariant is also complete. 

Note, the effectiveness of the approach depends on the nature of the given
invariant. For instance, if the given invariant was `(g = -1) $\vee$ (g =
lastN!)', which is also technically correct, then the theorem prover may
not be able to establish that in Lines~8 and~12 the variable `g' always
stores the same expression in n. However, it is our claim that in fact it is the
invariant `(g = -1) $\vee$ (g = lastN * \emph{factCache}(lastN-1))' that is
easier to infer by a human or by a potential tool, as justified by us two
paragraphs above.

\subsection{Salient Aspects Of Our Approach}

This paper makes two significant contributions. First, it tackles the
circularity problem that arises due to the use of a presumed-to-be OP
procedure in assertions and invariants and the use of these invariants in
proving the procedure to be OP. This requires us to prove the soundness of
an approach that verifies observational purity as well the validity of
invariants simultaneously (as they cannot be decoupled).

Secondly, as we show, a direct approach to this verification problem (which we
call the existential approach) reduces it to a problem of verifying that a logical formula
is a tautology. The structure of the generated formula, however, makes the resulting
theorem prover instances hard. We show how a conservative approximation can be
used to convert this hard problem into an easier problem of checking satisfiability
of a quantifier-free formula, which is something within the scope of state-of-the-art theorem
provers.

%% should I talk about why F(2) *3 instead of F(3) ?

The most closely related previous approaches are by Barnett et
al.~\cite{barnett200499,barnett2006allowing}, and by
Naumann~\cite{naumann2007observational}.  These approaches check observational
purity of procedures that maintain mutable global state. However, none of
these approaches use a function symbol in place of recursive calls or
within invariants. Therefore, it is not clear that these approaches can
verify recursive procedures. 
Barnett et al., in fact, state ``there is a circularity - it would take a delicate argument, and additional conditions,
to avoid unsoundness in this case''.
To the best of our knowledge ours is the first paper to show that it is
feasible to check observational purity of procedures that maintain mutable
global state for optimization purposes and that make use of recursion.


Being able to verify that a procedure is OP has many potential
applications. The most obvious one is that OP procedures can be
memoized. That is, input-output pairs can be recorded in a table, and calls
to the procedure can be elided whenever an argument is seen more than
once. This would not change the semantics of the overall program that calls
the procedure, because the procedure always returns the same value for the
same argument (and mutates only private global variables). Another
application is that if a loop contains a call to an OP procedure, then the
loop can be parallelized (provided the procedure is modified to access and
update its private global variables in a single atomic operation).

%In this paper, we do make a number of assumptions. The global variables
%that the procedure reads or updates are assumed to be private to it, in the
%sense that other code cannot access these variables at all.  For simplicity
%of presentation, we assume that loops are not present, and have been
%converted to recursion. However, we do not restrict the procedure to use
%only tail recursion. We do not as yet address pointers and memory
%allocation. 

The rest of this paper is structured as
follows. Section~\ref{sec:background} introduces the core programming
language that we address. Section~\ref{sec:semantics} provides formal
semantics for our language, as well as definitions of invariants and
observational purity. Section~\ref{sec:vcgen} describes our approach
formally. Section~\ref{sec:invariant} discusses an approach for generating
an invariant automatically in certain cases. Section~\ref{sec:experiments}  describes evaluation of our approach on a few realistic examples.
Section~\ref{sec:related} describes related work.


\newcommand{\elt}{\ensuremath{\in} }
\newcommand{\domain}[1]{#1}

\section{Language Syntax}
\label{sec:background}

In this paper, we assume that the input to the purity checker is a library consisting
of one or more procedures, with shared state consisting of one or more variables
that are private to the library. We refer to these variables as ``global'' variables to
indicate that they retain their values across multiple invocations of the library
procedures, but they cannot be accessed or modified by procedures outside
the library (that is, the clients of the library).

\begin{figure}[t!]
{\tt
\begin{tabular}{rll}
L \elt & \domain{Lib} & ::= $\overline{\code{g := c}}$ $\overline{\code{P}}$ \\
P \elt & \domain{Proc} & ::= p (x) \{ S; return y \} \\
S \elt & \domain{Stmt} & ::=  x := e | x := p(y) | S ; S | if (e) then S else S \\
e \elt & \domain{Expr} & ::= c | x | e op e | unop e \\
op \elt & \domain{Ops} & ::= + | - | / | * | \% | > | < | == | $\wedge$ | $\vee$ \\
unop \elt & \domain{UnOps} & ::= $\neg$ \\
\multicolumn{3}{c}{
x, y \elt  \domain{LocalId} $\cup$ \domain{GlobalId},
g \elt \domain{GlobalId},
% $\gvars \cup \lvars$ \\ 
c \elt $\vals$,
p \elt \domain{ProcId}
}
\end{tabular}
}
\caption{Programming language syntax and meta-variables}
 \label{fig:grammar}
\end{figure}

In Fig.~\ref{fig:grammar}, we present the syntax of a simple
programming language that we address in this paper.
Given the foundational focus of this work, we keep the programming
language very simple, but the ideas we present can be generalized.
%Most of the statements are standard statements borrowed
%from imperative languages. The rest of them are explained next.
A \code{return} statement is permitted only as the last statement of
the procedure.
%The language does not contain a \code{return} statement.
%At the end of execution of the procedure body, the procedure returns
%the final value of a special variable (indicated in the procedure declaration).
The language does not contain any looping construct.
Loops can be modelled as recursive procedures.
The formal parameters of a procedure are readonly and cannot be
modified within the procedure.
We omit types from the language. We permit only variables of primitive types.
In particular, the language does not allow pointers or dynamic memory allocation.
Note that expressions are pure in this language, and a procedure call
is not allowed in an expression. Each procedure call is modelled as a
separate statement.
%The
%statement `\code{havoc x}' assigns any value from the domain of the variable
%`\code{x}' to variable `\code{x}'.
%% to variable `x' from the declared domain of `x'
%% And the statement `assume x' allows the execution of the program to
%% proceed if the predicate `x' holds at the program point, otherwise it
%% halts the execution.

For simplicity in notation, we assume, without loss of generality,
that the library consists of a single procedure, with a single formal
parameter. We will abuse terminology and use the terms procedure and library interchangeably.
In the sequel, we will use the symbol $\proc$ (as a metavariable) to represent the procedure/library,
while we use $\procname$ (as a metavariable) to represent the \emph{name} of the procedure.
If the procedure of the form ``\code{p (n) \{ S; return r \}}'', we refer to \code{r} as the \emph{return}
variable and we refer to ``\code{S; return r}'' as the \emph{procedure body} and denote it $\text{body}(\proc)$.
The library also includes a sequence of initializing declarations of globals of the form ``\code{g := c}'', indicating
that \code{g} is initialized to the value \code{c} when execution begins.
%, and a single global variable.

 Finally, a note about terminology: throughout this paper we
use the word `procedure' to refer to the code procedure being analyzed, and use the word
`function' to refer to a mathematical function.


\newcommand{\vals}{\mathcal{V}}
\newcommand{\gvars}{G}
\newcommand{\lvars}{L}
\newcommand{\gmap}{\rho_g}
\newcommand{\gmaps}{\Sigma_G}
\newcommand{\lmap}{\rho_\ell}
\newcommand{\lmaps}{\Sigma_L}
\newcommand{\cstates}{S}
\newcommand{\cont}{s}
\newcommand{\lstack}{\gamma}
\newcommand{\initstate}{\sigma_{\text{init}}}

\newcommand{\initial}[1]{\textit{initial}(#1)}
\newcommand{\final}[1]{\textit{final}(#1)}
\newcommand{\inputval}[1]{\textit{input}(#1)}
\newcommand{\outputval}[1]{\textit{output}(#1)}

\newcommand{\extmap}{\sigma_e}


\newcommand{\iosem}[1]{\rightsquigarrow_{#1}}
\newcommand{\iosemP}{\iosem{\proc}}
\newcommand{\sssem}[1]{\rightarrow_{#1}}
\newcommand{\sssemP}{\sssem{\proc}}

\newcommand{\evalsto}[4]{ ({#1} \uplus {#2},{#3}) \Downarrow {#4}}

\newcommand{\labrule}[3]{\inferrule*[Lab={[#1]}]{#2}{#3}}

\section{A Semantic Definition of Purity}

In this section, we formalize the input-output semantics of a procedure $\proc$ as a relation $\iosem{\proc}$,
where $n \iosem{\proc} r$ indicates that an invocation of $\proc$ with input $n$ may return a result of $r$.
The procedure is then defined to be observationally pure if the relation $\iosem{\proc}$ is a (partial) function:
that is, if  $n \iosem{\proc} r_1$ and $n \iosem{\proc} r_2$, then $r_1 = r_2$.

The object of our analysis is a single procedure or, more generally, a collection of procedures (a library),
and not a whole program. The result of our analysis is valid for any program that uses the procedure/library.
The only assumptions we make are: (a) The shared state used by the library (the global variables) are private
to the library and cannot be modified by the rest of the program, and (b) The client invokes the library
procedures sequentially: no concurrent or overlapping invocations of the library procedures by a concurrent
client are permitted.

The following semantic formalism is motivated by the above observations. It can be seen as the semantics
of the so-called ``most general sequential client'' of procedure $\proc$, which is the program:
\code{while (*) { x = \proc(random()); }}.
The executions (of $\proc$) produced by this program include all executions (of $\proc$)  produced by all
sequential clients.

Let $\gvars$ denote the set of global variables. Let $\lvars$ denote the set of local variables.
Let $\vals$ denote the set of numeric values (that the variables can take).
An element $\gmap \in \gmaps = \gvars \hookrightarrow \vals$ maps global variables to their values.
An element $\lmap \in \lmaps = \lvars \hookrightarrow \vals$ maps local variables to their values.
Let $\cstates$ represent the set of statements. We use an element $\cont \in \cstates$ to represent
what remains to be executed in the procedure (and can, hence, be thought of as a program-point
representing what is to be executed next).
The set of runtime states is defined to be $(\cstates \times \lmaps)^* \times \gmaps$, where
the first component represents a runtime stack, and the second component the values of global
variables. The runtime stack is a sequence, each element of which is a pair consisting of the
remaining procedure fragment to be executed and the values of local variables.

We say that a state $\sigma$ is an \emph{entry-state} if its location is at the procedure entry point,
and we say that it is an \emph{exit-state} if its location is at the procedure exit point.
A procedure $\proc$ determines a single-step execution relation $\sssem{\proc}$, where $\sigma_1 \sssem{\proc} \sigma_2$ indicates
that execution proceeds from state $\sigma_1$ to state $\sigma_2$ in a single step.
We encode invocations of the procedure using the same relation $\sssem{\proc}$: 
$\sigma \sssem{\proc} \sigma'$  holds if $\sigma$ is an exit-state, $\sigma'$ is an entry-state,
and $\sigma$ and $\sigma'$ agree on the values of all global variables (but the value of the formal
parameter $n$ may be arbitrary in $\sigma'$ to indicate an invocation of the procedure with that
parameter value).

\begin{figure}
\begin{mathpar}
\labrule{assgn}{
\code{x} \in \lvars \\
\evalsto{\lmap}{\gmap}{\code{e}}{v}
}{
((\code{x := e; S}, \lmap) \lstack, \gmap)
\sssemP 
((\code{S}, \lmap[\code{x} \mapsto v]) \lstack, \gmap)
}

\labrule{assgn}{
\code{x} \in \gvars \\
\evalsto{\lmap}{\gmap}{\code{e}}{v}
}{
((\code{x := e; S}, \lmap) \lstack, \gmap)
\sssemP 
((\code{S}, \lmap) \lstack, \gmap[\code{x} \mapsto v])
}

\labrule{seq}{}{
(((\stmtSA ; \stmtSB) ; \stmtSC , \lmap) \lstack, \gmap) 
\sssemP
((\stmtSA ; (\stmtSB ; \stmtSC) , \lmap) \lstack, \gmap) 
}

\labrule{if-true}{
\evalsto{\lmap}{\gmap}{\expr}{\code{true}}
}{
( (\code{(if (\expr) then \stmtSA else \stmtSB); \stmtSC}, \lmap) \lstack, \gmap)
\sssemP
( (\code{\stmtSA; \stmtSC}, \lmap) \lstack, \gmap)
}

\labrule{if-false}{
\evalsto{\lmap}{\gmap}{\expr}{\code{false}}
}{
( (\code{(if (\expr) then \stmtSA else \stmtSB); \stmtSC}, \lmap) \lstack, \gmap)
\sssemP
( (\code{\stmtSB; \stmtSC}, \lmap) \lstack, \gmap)
}

\labrule{call}{
\evalsto{\lmap}{\gmap}{\code{e}}{v} \\
\proc = \code{p(n) \stmtSA}
}{
((\code{y = p(e); \stmtSB}, \lmap) \lstack, \gmap)
\sssemP
((\stmtSA, [n \mapsto v]) (\code{y = p(e); \stmtSB}, \lmap) \lstack, \gmap)
}

\labrule{return}{
\evalsto{\lmap}{\gmap}{r}{v}
}{
((\code{return r}, \lmap) (\code{y = p(e); \stmt}, \lmap') \lstack, \gmap)
\sssemP
(\stmt, \lmap'[ \code{y} \mapsto v]) \lstack, \gmap)
}

\labrule{top-level-call}{
\proc = \code{p(n) \stmtSA} \\
v \in \vals
}{
([], \gmap)
\sssemP
([(\stmtSA, [n \mapsto v])], \gmap)
}

\end{mathpar}
\caption{A small-step operational semantics for our language.}
\label{fig:semantics}
\end{figure}

Note that all the following definitions are parametric over a given procedure $\proc$.
(Thus, what we refer to as an execution below is shorthand for ``execution of $\proc$''.

We define an \emph{execution} to be a sequence of states $\sigma_0 \sigma_1 \cdots \sigma_n$ such that
$\sigma_i \sssem{\proc} \sigma_{i+1}$ for all $0 \leq i < n$.
%(In this paper we consider only \emph{sequential} executions of the procedure. Thus, no concurrent invocations
%of the procedure by a concurrent client of the library is allowed.)
Let $\initstate$ denote the \emph{initial state} of the library.
We say that a state $\sigma$ is \emph{ reachable} if there exists an execution $\pi = \initstate \sigma_1 \cdots \sigma$.

We define a \emph{history} (of $\proc$) to be an execution $\pi = \sigma_0 \sigma_1 \cdots \sigma_n$ such that
(a) $\sigma_0$ is the initial state of the library,
(b) $\sigma_n$ is an exit-state, and
(c) $\pi$ has an equal number of entry-states and exit-states.
Note that a history represents a complete  execution produced by a
sequence of invocations of the procedure.

We say that an execution is a \emph{trace} (of $\proc$) if it represents the complete execution
of the procedure produced by a single invocation of the procedure.
Note, however, that a trace may contain nested sub-traces,
which are themselves traces, due to recursive calls.
Given a trace $\pi = \sigma_0 \cdots \sigma_n$, we define
$\initial{\pi}$ to be $\sigma_0$,
$\final{\pi}$ to be $\sigma_n$,
$\inputval{\pi}$ to be value of the input parameter in $\initial{\pi}$,
and $\outputval{\pi}$ to be the value of the return variable in $\final{\pi}$.

We define the relation $\iosemP$ to be $\{ (\inputval{\pi},\outputval{\pi}) \; | \; \pi \text{ is a trace of } \proc \}$.
% we say $n \iosemP r$ if there exists a reachable exit state where the value of formal
% parameter is $n$ and return value is $r$.

\begin{definition}[Observational Purity]
A procedure $\proc$ is said to be \emph{observationally pure} if the relation $\iosem{\proc}$ is a (partial) function:
that is, if for all $n$, $r_1$, $r_2$, if  $n \iosem{\proc} r_1$ and $n \iosem{\proc} r_2$, then $r_1 = r_2$.
\end{definition}

\subsection*{Invariants}

We now formalize the concept of an \emph{invariant}. Let $\varphi$ denote a boolean-valued expression over
the set of global variables. We write $\gmap \models \varphi$ to denote that $\varphi$ evaluates to true in a state
where the globals have values as in $\gmap$.

However, as we saw with our running example, it is useful to allow the use of the name of a procedure in
an invariant: \eg, an invariant such as ``$(g == -1) \vee (g == \text{factorial}(3))$''.
We interpret such extended invariants as below.
Let $\extmap$ denote an extended map that assigns global variables a (numeric) value and assigns the
procedure names a mathematical function as a value. We can then extend the expression evaluation semantics
in a straightforward fashion to evaluate extended boolean expressions of the above form. 
We will write $\extmap \models \varphi_e$ to denote that the extended boolean expression $\varphi_e$
evaluates to true under $\extmap$.

Given a global state $\gmap$, a procedure name $\proc$, and a mathematical function $f$, we will write
$\gmap[\proc \mapsto f]$ to indicate the extended map that assigns $\proc$ the value $f$ and every
global $x$ the value $\gmap[x]$.

%An invariant is essentially a logical formula that holds true
%at entry to and exit from the procedure. 

We say that a $\gmap \in \gmaps$ is a \emph{reachable global entry-state} of $\proc$ if
there is a reachable entry-state of the form $(\lstack,\gmap)$.

\begin{definition}[Invariant]
Given a procedure $\proc$ and a mathematical function $f$, we say that $(\proc,f)$ satisfy
an invariant $\inv$ if for every reachable global entry-state $\gmap$ of $\proc$,
$\inv$ holds in $\gmap[\proc \mapsto f]$: that is, if $\gmap[\proc \mapsto f] \models \inv$.
We say that procedure $\proc$ satisfies an invariant $\inv$ if there exists a (mathematical)
function $f$ such that $(\proc,f)$ satisfies $\inv$.
\end{definition}
\newcommand{\existformula}{\psi^e}
\newcommand{\EA}{\textsc{ea}}
\newcommand{\IW}{\textsc{iw}}

\newcommand{\initformula}{\logicalformula{init}}

\section{Checking Purity Using a Theorem Prover}
\label{sec:vcgen}

In this section we provide two different approaches that, given a procedure
$\proc$ and a candidate invariant $\inv$, use a theorem prover to check
conservatively whether procedure $\proc$ satisfies $\pureinv$.



\subsection{Verification Condition Generation}

We first describe an adaptation of standard verification-condition generation
techniques that we use as a common first step in both our approaches.
Given a procedure $\procP$, a candidate invariant $\inv$, our goal is to compute a
% the function $\mainvc{\procP}{\inv}$ returns a
pair $(\post,\vc)$ where $\post$ is a postcondition describing the state that exists after an execution of
$\text{body}(\procP)$ starting from a state that satisfies $\inv$, and $\vc$ is a verification-condition that must hold true
for the execution to satisfy its invariants and assertions.

We first transform the procedure body as below to create an internal representation that is input to the
postcondition and verification condition generator. In the internal representation, we allow the following
extra forms of statements (with their usual meaning): \code{havoc(x)}, \code{assume e}, and  \code{assert e}.
\begin{enumerate}
\item For any assignment statement ``\code{x := e}'' where \code{e} contains \code{x}, we introduce a new temporary
variable \code{t} and replace the assignment statement with ``\code{t := e; x := t}''.
\item For every procedure invocation ``\code{x := $\procname$(y)}'', we first ensure that \code{y} is a local variable (by introducing
a temporary if needed). We then replace the statement by the code fragment
``\code{assert $\inv$; havoc(g1); ... havoc(gN); assume $\inv \wedge$ x = $\procname$(y)}'',
where \code{g1} to \code{gN} are the global variables.

Note that the
function call has been eliminated, and replaced with an ``assume''
expression that refers to the function symbol $\procname$. In other words,
there are no function calls in the transformed procedure.
\item We replace the ``\code{return x}'' statement by ``\code{assert $\inv$}''.
\end{enumerate}
Let $\tbody(\proc, \inv)$ denote the transformed body of procedure $\proc$ obtained as above.

%We use an auxiliary function $\auxvcfn$ which is similar to $\mainvcfn$, but accepts a statement $\stmtS$ as
%parameter instead of a procedure, and it has an extra parameter $\pre$ that is a precondition describing
%the state before the execution of the statement.
% Fig.~\ref{fig:vcgen} contains a definition of $\mainvc{\procP}{\inv}$  and  $\auxvc{\stmtS}{\inv}{\pre}$.
% contains a definition of the functions to compute the postcondition and verification condition.

\begin{figure}
\[
\begin{array}{ll}
\postfn(\pre, \code{x := e}) &= (\exists \code{x}. \pre) \wedge (\code{x =
  e}) \ (\text{if } \code{x} \not\in \text{vars}(\code{e})) \\
\postfn(\pre, \code{havoc(x)}) &= \exists \code{x}. \pre \\
\postfn(\pre, \code{assume e}) &= \pre \wedge \code{e} \\
\postfn(\pre, \code{assert e}) &= \pre \\
\postfn(\pre, \stmtSA ; \stmtSB) &= \postfn( \postfn(\pre, \stmtSA), \stmtSB) \\
\multicolumn{2}{l}{\postfn(\pre, \code{if \expr{} then \stmtSA{} else \stmtSB{}}) = \postfn(\pre \wedge \expr, \stmtSA) \vee \postfn(\pre \wedge \neg \expr, \stmtSB)} \\
\\
\vcfn(\pre, \code{assert e}) &= (\pre \Rightarrow e) \\
\vcfn(\pre, \stmtSA ; \stmtSB) &= \vcfn(\pre, \stmtSA) \wedge \vcfn( \postfn(\pre, \stmtSA), \stmtSB) \\
\multicolumn{2}{l}{
\vcfn(\pre, \code{if \expr{} then \stmtSA{} else \stmtSB{}}) = \vcfn(\pre \wedge \expr, \stmtSA) \wedge \vcfn(\pre \wedge \neg \expr, \stmtSB)
} \\
\vcfn(\pre, \stmt) &= \text{true} (\text{for all other \stmt}) \\
\\
\mainvc{\procP}{\inv} = (\postfn(&\inv, \tbody(\proc,\inv)), \vcfn(\inv,
\tbody(\proc,\inv)) \wedge (\initstatefn(\proc) \Rightarrow \inv))
\end{array}
\]
\caption{Generation of verification-condition and postcondition.}
\label{fig:vcgen}
\end{figure}

We then compute postconditions as formally described in Fig.~\ref{fig:vcgen}.
This lets us compute for each program point $\ell$ in the procedure,
a condition $\varphi_{\ell}$ that describes what we expect to hold true when execution reaches $\ell$ if we start
executing the procedure in a state satisfying $\inv$ and if every recursive invocation of the procedure also
terminates in a state satisfying $\inv$. We compute this using the standard rules for the postcondition of a statement.
%A recursive call \code{x = p(y)}  is modelled by the \emph{assumed} specification for the procedure, namely that
%the global variables may be modified in any way, but the state at the return site will satisfy $\inv$
%and the returned value will equal $p(y)$. (Note that in a logical formula the symbol $p$ is used to represent the
%mathematical function realized by procedure $p$, assuming $p$ to be pure.)
%
For an assignment statement ``\code{x := e}'', we use existential quantification over \code{x} to represent the value
of \code{x} prior to the execution of the statement. If we rename these existentially quantified variables with unique new
names, we can lift all the existential quantifiers to the outermost level. When transformed thus, the condition $\varphi_{\ell}$
takes the form $\exists x_1 \cdots x_n. \varphi$, where $\varphi$ is quantifier-free and $x_1, \cdots, x_n$ denote
intermediate values of variables along the execution path from procedure-entry to program point $\ell$.

We compute a verification condition $\vc$ that represents the conditions we must check to ensure that
an execution through the procedure satisfies its obligations: namely, that the invariant holds true at every call-site
and at procedure-exit. Let $\ell$ denote a call-site or the procedure-exit. We need to check that $\varphi_{\ell} \Rightarrow \inv$
holds. Thus, the generation verification condition essentially consists of the conjunction of this check over all call-sites
and procedure-exit.

Finally, the function $\mainvcfn$ computes the postcondition and verification condition for the entire procedure as shown
in Fig.~\ref{fig:vcgen}. Note that this adds the check that the initial state too must satisfy $\inv$ as the basis condition for
induction. $\initstatefn(\proc)$ is basically the formula  ``\code{g1 = c1}
$\wedge \ldots$ \code{gN = cN}'' (see Section~\ref{sec:background}).
% (\initformula \Rightarrow \inv 

\subsection{Our intermediate representation}
\label{sec:intermediate}
We transform the given procedure into the following representation
in-order to encode it in logic. For example procedure factorial
in Listing~\ref{lst:factorialTransformed}, is the transformed version
of procedure factorial in Listing~\ref{lst:factorialSimple}.

Our analysis expects the following :
\begin{enumerate}
\item Procedure calls are approximated using function symbols. The
  statement `x = foo(y)' is replaced with `x = $\F$(y)'. Since
  procedure calls may modify global variables, we add the statement
  `havoc g' for each global variable `g' accessible from the
  procedure.
\item Next, the procedure must have extra variables to store the value
  of global variables at procedure boundaries. Before the
  $\mathit{i^{th}}$ procedure call, we add the statement `gbef$_i$ =
  g' and after it we add `gaft$_i$ = g'. Similarly, we add the
  statement `gout = g' after the assignment to the variable `\retVar'.
\item The procedure should be in static single assignment (SSA)
  form. The procedure should be converted to SSA after the above
  mentioned points have been satisfied.
\end{enumerate}

\begin{lstlisting}[language=c, caption= {Procedure factorial from
      Listing~\ref{lst:factorialSimple} converted to the form our
      approach expects. We refer to this procedure as `transformed
      factorial'.}, label=lst:factorialTransformed]
int g = -1;
int transformedFactorial( int n) { // redo
  if(n <= 1) {
    retVar = 1;
    gout = g;
  } else if(g == -1 && n == 19) {
    gbef1 = g;
    temp1 = F(18);  // temp1 = factorial(18)
    havoc(g1);
    gaft1 = g1;
    g2 = 19 * temp1;
    retVar = g2;
    gout = g2;
  } else if (g != -1 && n == 19) {
    retVar = g;
    gout = g;
  } else {
    gbef2 = g;
    temp2 = F( n - 1 );  //temp2 = factorial(n-1)
    havoc(g2);
    gaft2 = g3;
    retVar = n * temp2;
    gout = g3;
  }
}
\end{lstlisting}

%% In procedure `transformedFactorial'
%% Listing~\ref{lst:factorialTransformed}, in comparison to procedure
%% factorial in Listing~\ref{lst:factorialSimple}, the return statement
%% (line 4) is replaced by an assignment to variable `\retVar' (line
%% 4). After line 5 of `transformed factorial', an extra variable `gout'
%% is assigned the value of global variable `g' (value of `g' at end of
%% program).  Similarly, variable `gbef1' (line 7, `transformed
%% factorial') is added to capture the value of the global variable
%% before the procedure call (a program boundary), and variable `gaft1'
%% is inserted at line 11 to capture the value of `g' after the procedure
%% call. Also, havoc statements at line 8, 20 and 9, 21 over-approximate
%% the return from the procedure call statement and updates to the global
%% variable respectively. The procedure call statement is substituted
%% with function symbols in line 10 and 22, accounting the given
%% procedure as a function.

Now we compare `transformed factorial' and procedure factorial from
Listings~\ref{lst:factorialTransformed} and \ref{lst:factorialSimple}
respectively. The return statement (line 4) in procedure factorial
is replaced by assignment to variable `\retVar' (line 4). Also, we
have a added an assignment statement (line 5) in `transformed
factorial', that defines variable `gout' in-order to store the value
of variable `g' at the procedure boundary. Similarly, variable `gbef1'
(line 7) is added in `transformed factorial' to capture the value of
the global variable before the procedure call and assignment to
variable `gaft1' is inserted at line 10 to capture the value of `g'
after the procedure call. Also, havoc statements at lines 9 and 20
over-approximate any side effects to the global variables. And in
lines 8 and 19, the procedure call statements are substituted for
function symbols.

\subsection{Converting the program into logical formulae} \label{sec:approaches}

\subsection{Path condition generation}\label{sec:vcgen}
For our analysis we represent the given procedure in logic, such that
the representation captures the argument value at the beginning of the
procedure, the return value at the end of the procedure and the values
of the global variables at all the boundaries of the procedure. We
track the values of the global variables using extra variables as
explained in Section~\ref{sec:intermediate}. Thus, the set of free
variables in a path condition of a procedure are $X : \mi{\{n, gin,
  gout, g, \gbef_1 \cdots \gbef_m, \gaft_1 \cdots \gaft_m,}$ \\ ${\retVar, \F
  \}}$ all the remaining variables are universally quantified.

\begin{figure}
  \begin{align*}
    \pathCondition :=
    &(n \leq 1 \wedge retVar = 1 \wedge gout = g) \vee \\
    &(n > 1 \wedge g = -1 = gbef \wedge n = 19 \wedge temp1 = \F(18) \\
    \;&\wedge gaft1 = g1 \wedge g2 = 19 * temp1
    \wedge retVar = g2) \vee\\
    &(n > 1 \wedge \neg( g = -1 \wedge n = 19) \wedge g \neq -1
    \wedge n = 19 \wedge retVar = g = gout) \vee\\
    &(n > 1 \wedge n \neq 19 \wedge gbef2 = g \wedge temp2 = \F( n
    - 1) \wedge gaft2 = g1\\
    &\wedge retVar = n * temp2 \wedge gout = g1)\\
  \end{align*}
  \caption{Formula representing procedure `transformed factorial' in
    Listing~\ref{lst:factorialTransformed} (assuming that function
    $\F$ is equivalent to procedure factorial).}
  \label{fig:pathCondition}
\end{figure}

For example, the procedure `transformed factorial' in
Listing~\ref{lst:factorialTransformed} is expressed in logic as shown
in Figure~\ref{fig:pathCondition}. Each disjunct in
Figure~\ref{fig:pathCondition} represents a straight-line execution of
procedure factorial (from beginning, until end) . For instance, $(n <=
1 \wedge retVar = 1 \wedge gout = g)$ represents the case where $`n
\leq 1'$.

Representation of a program in logic is straight forward once it is
converted to our intermediate representation in
Section~\ref{sec:intermediate}. All the standard imperative statements
become conjuncts in the formula. `havoc' statements are omitted.
`assume x' are replaced with a conjunct `x' in the formula.

\subsection{Approach 1: Existential Approach}

Let $\proc$ be a procedure with input parameter $n$ and return variable $r$.
Let\\ $\mainvc{\proc}{\inv}$ = $(\post,\vc)$.
Let $\existformula$ denote the formula $\vc \wedge (\post \Rightarrow (r = p(n)))$.
Let $\overline{x}$ denote the sequence of all free variables in $\existformula$ except for $p$.
We define $\EA(\proc,\inv)$ to be the formula $ \forall \overline{x}. \existformula$.

In this approach, we use a theorem prover to check whether $\EA(\proc,\inv)$ is satisfiable.
As shown by the following theorem, satisfiability of $\EA(\proc,\inv)$ establishes that $\proc$
satisfies $\pureinv$.

\begin{theorem}
\label{theorem:EA}
A procedure $\proc$ satisfies $\pureinv$ if
$\exists p. \EA(\proc,\inv)$ is a tautology
(which holds iff $\EA(\proc,\inv)$ is satisfiable).
\end{theorem}

\begin{proof}
Note that $p$ is the only free variable in $\EA(\proc,\inv)$. Assume that $[p \mapsto f]$ is a
satisfying assignment for $\EA(\proc,\inv)$.
We prove that for every trace $\pi$ the following hold:
(a) $\outputval{\pi} = f(\inputval{\pi})$ and
(b) If $(\initial{\pi},f)$ satisfies $\inv$, then $(\final{\pi},f)$ also satisfies $\inv$.

The proof is by contradiction. Let $\pi$ be the shortest trace that does not satisfy at least
one of the two conditions (a) and (b).

Let us first consider a trace $\pi$ without any sub-traces (\ie, without a procedure call).
Consider any transition $\sigma_i \sssemP \sigma_{i+1}$ in $\pi$
caused by an assignment statement $\stmt$. Our verification-condition generation produces
a precondition $\pre_{\stmt}$ and a postcondition $\post_{\stmt}$ for $\stmt$. Further, this generation
guarantees that if $(\sigma_i,f)$ satisfies $\pre_{\stmt}$, then $(\sigma_{i+1},f)$ satisfies $\post_{\stmt}$.
This lets us prove (via induction over $i$), that if $(\initial{\pi},f)$ satisfies $\inv$,
then $(\sigma_{i+1},f)$ satisfies $\post_{\stmt}$.

We now consider a trace $\pi$ that contains a sub-trace $\pi' = \sigma_i \cdots \sigma_j$ corresponding to a procedure call
statement $\stmt$. Our verification-condition generation produces a precondition $\pre_{\stmt}$,
a postcondition $\post_{\stmt}$  and a conjunct $\pre_{\stmt} \Rightarrow \inv$ in $\vc$ for $\stmt$.
We extend the induction over $i$ to handle recursive calls as below.
Our inductive hypothesis guarantees that $(\sigma_i,f)$ satisfies $\pre_{\stmt}$.
Further, we know that $[p \mapsto f]$ is a satisfying assignment for $\EA(\proc,\inv)$,
which includes the conjunct $\pre_{\stmt} \Rightarrow \inv$. Hence, it follows that  $(\sigma_i,f)$ satisfies $\inv$.
Since $\pi'$ is a shorter trace than $\pi$, we can assume that it satisfies conditions (a)
and (b). This is sufficient to guarantee that $(\sigma_j,f)$ satisfies $\post_{\stmt}$.

Thus, we can establish $(\final{\pi},f)$ satisfies $\post$ and $\inv$. Further, since 
$\EA(\proc,\inv)$ includes the conjunct $\post \Rightarrow (r = p(n))$, it follows that
$\outputval{\pi} = f(\inputval{\pi})$.
\end{proof}

\subsection{Approach 2: Impurity Witness Approach}

The existential approach presented in the previous section has a drawback. Checking satisfiability of $\EA(\proc,\inv)$
is hard because it contains universal quantifiers and existing theorem provers do not work well enough for this
approach. We now present an approximation of the existential approach that is easier to use with existing theorem
provers. This new approach, which we will refer to as the impurity witness approach, reduces the problem to
that of checking whether a quantifier-free formula is unsatisfiable, which is better suited to the capabilities of
state-of-the-art theorem provers. This approach focuses on finding a counterexample to show that the
procedure is impure or it violates the candidate invariant.

Let $\proc$ be a procedure with input parameter $n$ and return variable $r$.
Let $\mainvc{\proc}{\inv}$ = $(\post,\vc)$.
Let $\post_\alpha$ denote the formula obtained by replacing every free variable $x$ other than $p$ in $\post$
by a new free variable $x_\alpha$. Define $\post_\beta$ similarly.
Define $\IW(\proc, \inv)$ to be the formula $(\neg \vc) \vee (\post_\alpha \wedge \post_\beta \wedge (n_\alpha = n_\beta) \wedge (r_\alpha \neq r_\beta))$.

The impurity witness approach checks whether $\IW(\proc, \inv)$ is satisfiable. This can be done by separately checking
whether $\neg \vc$ is satisfiable and whether $(\post_\alpha \wedge \post_\beta \wedge (n_\alpha = n_\beta) \wedge (r_\alpha \neq r_\beta))$
is satisfiable. As formally defined, $\vc$ and $\post$ contain embedded existential quantifications. As explained earlier,
these existential quantifiers can be moved to the outside after variable renaming and can be omitted for a satisfiability check.
(A formula of the form $\exists \overline{x}. \psi$ is satisfiable iff $\psi$ is satisfiable.)
As usual, these existential quantifiers refer to intermediate values of variables along an execution path.
Finding a satisfying assignment to these variables essentially identifies a possible execution path (that
satisfies some other property).

\begin{theorem}
A procedure $\proc$ satisfies  $\pureinv$ if $\IW(\proc, \inv)$ is unsatisfiable.
\end{theorem}

\begin{proof}
We prove the contrapositive.

We say that a pair of feasible executions $(\pi_1, \pi_2)$ is an impurity witness if there is a trace
$\pi_a$ in $\pi_1$ and a trace $\pi_b$ in $\pi_2$ such that $\pi_a$ and $\pi_b$ have the same input
value but different return values. Otherwise, we say that $(\pi_1, \pi_2)$ is pure. We extend
this notion and say that a single execution $\pi$ is pure if $(\pi,\pi)$ is pure.

We say that a function $f$ is \emph{compatible} with a set of executions $\Pi$ if for every trace
$\pi \in \Pi$, $\outputval{\pi} = f(\inputval{\pi})$.

We say that a pure feasible execution $\pi$ is a $\inv$-violation witness if there is some function $f$
that is compatible with $\{\pi\}$ such that $(\pi,f) \nvDash \inv$. Otherwise, we say that $\pi$ satisfies
the invariant. Note that this definition introduces a conservative approximation as we explain later.
% Otherwise, we say that $\pi$ is an invariant violation witness.

We will consider \emph{minimal} witnesses of the following form.
We say that an impurity witness $(\pi_1, \pi_2)$ is minimal if for every proper prefix $\pi_1'$ of
$\pi_1$ and every proper prefix $\pi_2'$ of $\pi_2$, the following hold:
(a) $(\pi_1',\pi_2)$ is pure,
(b) $(\pi_1,\pi_2')$ is pure,
(c) $\pi_1'$ satisfies the invariant, and
(d) $\pi_2'$ satisfies the invariant.

We say that an $\inv$-violation witness $\pi$ is minimal if no proper prefix $\pi'$ of
$\pi$ is a $\inv$-violation witness.

If $\proc$ does not satisfy $\pureinv$, then there exists a minimal impurity witness
or a minimal $\inv$-violation witness. Note that our definition of $\inv$-violation witness is
a conservative approximation and the converse of the preceding claim does not hold.
A  $\inv$-violation witness does not mean that $\proc$ does not satisfy $\pureinv$.

In the first case, $(\post_\alpha \wedge \post_\beta \wedge (n_\alpha = n_\beta) \wedge (r_\alpha \neq r_\beta))$
must be satisfiable (as we show below). In the second case, $\neg \vc$ must be satisfiable (as we show below).
Thus,  $\IW(\proc, \inv)$  is satisfiable in either case. The theorem follows.

We establish the above result as below.
% using the following lemmas.
If we have a trace that satisfies the invariant and the set of its sub-traces are pure,
then the valuations assigned to variables by the trace satisfies $\post$.
Thus, if $(\pi_1,\pi_2)$ are a minimal impurity witness, let $\pi_a$ and $\pi_b$
be the two traces in $\pi_1$ and $\pi_2$ that are incompatible.
We can assign values to variables in $\post_\alpha$ from $\pi_a$, and
assign values to variables in $\post_\beta$ from $\pi_b$ to get a satisfying
assignment for $(\post_\alpha \wedge \post_\beta \wedge (n_\alpha = n_\beta) \wedge (r_\alpha \neq r_\beta))$.

If $\pi$ is a minimal $\inv$-violation witness, let $\pi'$ be the trace that contains the invariant violation.
Assigning values to variables as in $\pi'$ produces a satisfying assignment for
% $\neg (\post \Rightarrow \inv)$ and, hence, for
$\neg \vc$.
\end{proof}

\section{Generating the invariant}
\label{sec:invariant}

We now describe a simple but reasonably effective semi-algorithm for
generating a candidate invariant automatically from the given
procedure. Our approach of Section~\ref{sec:vcgen} can work with a manually
provided invariant, or from the candidate invariant generated by this
semi-algorithm whenever it terminates. The approach is iterative, and
grows the candidate invariant in each iteration. The initial candidate
invariant $I_0$ just captures the initial values of the global variable. In
each iteration $k$, $\postfn(I_{k-1}, \proc)$ is first computed. Then, the
pre-conditions that are computed by $\postfn$ at the points before the
recursive call-sites and at the end of the procedure are disjuncted with
$I_{k-1}$ to yield $I_k$; also, outermost level universal quantifiers are
introduced in $I_k$ for all variables other than the global variables and
the procedure symbol $\procname$.  This is continued until two consecutive
iterations $m$ and $m+1$ are encountered such that $I_m \equiv I_{m+1}$.

In our running example, $I_0$ would be `g = -1'. $\postfn(I_0,
\mathrm{factCache})$ would compute $I_0$ itself as the pre-condition at the
point just before the recursive call-site, and `g = -1 $\vee$ g = lastN *
$\procname$(lastN-1)' (after certain simplifications) as the pre-condition
at the end of the
procedure. Therefore, $I_1$ would be `g = -1 $\vee$ g = lastN *
$\procname$(lastN-1)'. When we compute $\postfn(I_1, \proc)$, the
pre-conditions happen to be $I_1$ itself at both the program points
mentioned above. Therefore, the approach terminates with $I_1$ as the
candidate invariant.

% Note, this approach can go into non-termination in
% some cases, e.g., in the presence of statements such as `g = g + 1', where
% g is the private global variable of the procedure.

% \section{Older Formalism}

We make the following simplifying assumptions:
\begin{itemize}
%\item We analyze a single procedure in isolation and we assume that no
%  other procedure writes to the global variables accessed by procedure
%  \foo. In-case of multiple procedures, the invariant has multiple
%  function symbols, one for each procedure. 
\item All variables are defined before use, except global variables.
\item The variable assigned the return value in a procedure call
  statement is always a local variable.
\item There is at max one procedure call statement in every straight line
  execution of \foo from the beginning until it's
  end.
  \end{itemize}
%% Note : all of the above mentioned restrictions can be omitted. They
%% are for convenience.

\subsection{Our intermediate representation}
\label{sec:intermediate}
We transform the given procedure into the following representation
in-order to encode it in logic. For example procedure factorial
in Listing~\ref{lst:factorialTransformed}, is the transformed version
of procedure factorial in Listing~\ref{lst:factorialSimple}.

Our analysis expects the following :
\begin{enumerate}
\item Expressions do not have procedure call statements as
  sub-expressions.(omit? too much detail)
\item Procedure calls are approximated using function symbols. The
  statement `x = foo(y)' is replaced with `x = $\F$(y)'. Since
  procedure calls may modify global variables, we add the statement
  `havoc g' for each global variable `g' accessible from the
  procedure.
\item The input procedure has `m' procedure calls.
\item Next, the procedure must have extra variables to store the value
  of global variables at procedure boundaries. Before the
  $\mathit{i^{th}}$ procedure call, we add the statement `gbef$_i$ =
  g' and after it we add `gaft$_i$ = g'. Similarly, we add the
  statement `gout = g' after the assignment to the variable `\retVar'.
\item The procedure should be in static single assignment (SSA)
  form. The procedure should be converted to SSA after the above
  mentioned points have been satisfied.
\end{enumerate}

\begin{lstlisting}[language=c, caption= {Procedure factorial from
      Listing~\ref{lst:factorialSimple} converted to the form our
      approach expects. We refer to this procedure as `transformed
      factorial'.}, label=lst:factorialTransformed]
int g = -1;
int transformedFactorial( int n) { // redo
  if(n <= 1) {
    retVar = 1;
    gout = g;
  } else if(g == -1 && n == 19) {
    gbef1 = g;
    temp1 = F(18);  // temp1 = factorial(18)
    havoc(g1);
    gaft1 = g1;
    g2 = 19 * temp1;
    retVar = g2;
    gout = g2;
  } else if (g != -1 && n == 19) {
    retVar = g;
    gout = g;
  } else {
    gbef2 = g;
    temp2 = F( n - 1 );  //temp2 = factorial(n-1)
    havoc(g2);
    gaft2 = g3;
    retVar = n * temp2;
    gout = g3;
  }
}
\end{lstlisting}

%% In procedure `transformedFactorial'
%% Listing~\ref{lst:factorialTransformed}, in comparison to procedure
%% factorial in Listing~\ref{lst:factorialSimple}, the return statement
%% (line 4) is replaced by an assignment to variable `\retVar' (line
%% 4). After line 5 of `transformed factorial', an extra variable `gout'
%% is assigned the value of global variable `g' (value of `g' at end of
%% program).  Similarly, variable `gbef1' (line 7, `transformed
%% factorial') is added to capture the value of the global variable
%% before the procedure call (a program boundary), and variable `gaft1'
%% is inserted at line 11 to capture the value of `g' after the procedure
%% call. Also, havoc statements at line 8, 20 and 9, 21 over-approximate
%% the return from the procedure call statement and updates to the global
%% variable respectively. The procedure call statement is substituted
%% with function symbols in line 10 and 22, accounting the given
%% procedure as a function.

Now we compare `transformed factorial' and procedure factorial from
Listings~\ref{lst:factorialTransformed} and \ref{lst:factorialSimple}
respectively. The return statement (line 4) in procedure factorial
is replaced by assignment to variable `\retVar' (line 4). Also, we
have a added an assignment statement (line 5) in `transformed
factorial', that defines variable `gout' in-order to store the value
of variable `g' at the procedure boundary. Similarly, variable `gbef1'
(line 7) is added in `transformed factorial' to capture the value of
the global variable before the procedure call and assignment to
variable `gaft1' is inserted at line 10 to capture the value of `g'
after the procedure call. Also, havoc statements at lines 9 and 20
over-approximate any side effects to the global variables. And in
lines 8 and 19, the procedure call statements are substituted for
function symbols.

\subsection{Invariant}\label{sec:invariant}

\begin{definition}[invariant]
  Invariant $\inv$ represents the set of
  global states at the boundaries of a procedure. $\inv$ is a formula,
  with global variables accessed by the given procedure and a
  uninterpreted function symbol $\F$ as the set of free variables. The
  uninterpreted function symbol $\F$ is a placeholder for the
  mathematical function equivalent to the given procedure. The initial
  global state satisfies the invariant.
\end{definition}

The invariant is produced by syntactically analyzing the
program. Invariant is expressed in a logic with un-interpreted
functions. The uninterpreted function is used to abstract out the
recursive procedure calls. It is assumed that the uninterpreted
function symbol represents the same mathematical function as the given
procedure. For example for procedure factorial in
listing~\ref{lst:factorialSimple} we use the invariant $\mathit{g = -1
  \vee g = \F(2) * 3}$, where $\F(2)$ is assumed to be equal to $x$,
$x = \mi{factorial(2)}$.

\subsection{Semantics}

\begin{definition}[at-entry(history/trace/trace-segment)  $\rightarrow$
    value] 
  $\\\atEntry{x}$ denotes the value of global variable $\g$ at the
  starting point of $x$ where $x$ is a history, trace or trace
  segment.
\end{definition}

\begin{definition}[trace]
  A trace $\trace$ is a complete execution of the procedure starting
  from a given global state and a tuple of formal arguments. A trace
  upon completion gives an end state and a return value.\\ Traces may
  have sub-traces, which by definition are traces (recursive calls).
\end{definition}

\begin{definition}[history]
  A history $\history$ is a sequence of consecutive traces, with the
  first trace starting from a given global state, and each subsequent
  trace begins in the state in which the previous trace ends.
\end{definition}

\begin{definition}[histories$\mhyphen$of (global state)$\rightarrow$
    set of histories] 
  $\historyOf{\gmap}$ returns the set of all possible histories such
  that for each history $\history$ in the set, $\history$ is a history
  of procedure `\foo' with $\atEntry{\history} = \gmap$.
\end{definition}

\begin{definition}[traces$\mhyphen$Of(history) $\rightarrow$ set of
    traces] $\tracesOf{\history}$ returns the set of traces and
  sub-traces in history $\history$.
\end{definition}

Invariant $\inv$ is assumed at the beginning of the procedure and
asserted at the end. Also, $\inv$ is asserted before the procedure
call and assumed after the procedure call. Above mentioned treatment
is similar to handling of pre-conditions and post-conditions for
recursive procedures. For the invariant, it is assumed that the given
procedure is observationally pure. The invariant has the free variable
$\F$ which is also used to abstract over procedure calls and it is
defined only if the given procedure is observationally pure.

Let $\gmap$ be a global state, and $\F'$ be a function, and $\inv$
be a invariant. If we substitute $\gmap$ for $g$ and $\F'$ for $\F$
in $\inv$ and it is true, then we say that $(\gmap, \F')$ satisfies
$\inv$, which is denoted as $(\gmap, \F') \satisfies \inv$. For
instance let $\inv := g = -1 \vee g = \F(18) * 3$ then $(3!, \lambda
n. n!) \satisfies \inv$.

\subsection{Observational purity}\label{sec:op}

\begin{definition}[argument(trace $\rightarrow$ value)]
  $\param{\trace}$ denotes the mapping of the argument `$\n$' to its
  value in the beginning of $\trace$.
\end{definition}


\begin{definition}[observational purity(set of states)]
  A procedure `\foo' is observational pure wrt a given set of global
  states $\gmaps$ if for each history $\history$ of procedure
  `\foo', all pairs of traces in the history that begin in a global
  states $\gmap$ and $\gmap'$ respect each other where $\gmap
  \in \gmaps$, $\gmap' \in \gmaps$.  In other words
  $\forall \gmap \in \gmaps.\forall \, \history \in
  \historyOf{\gmap}. \forall \trace_1 \in
  \tracesOf{\history}. \forall \trace_2 \in
  \tracesOf{\history}.\\ \param{\trace_1} = \param{\trace_2} \implies
  \return{\trace_1} = \return{\trace_2}$.
\end{definition}

\begin{lstlisting}[caption={Procedure `remember' : always returns the
      argument from its first call}, label=lst:remember]
int g = 0;
int init = 0;
int remember( int n) {
  if(init == 0){
    init = 1;
    g = n;
  }
  return g;
}
\end{lstlisting}

In a history, if all traces return the same result for an argument
value, there exists a function $\F$ that respects the given
procedure. In a history, the given procedure mimics a unique function,
but the mimicked function varies across histories. If the invariant is
a set of concrete states then there exists a unique function that is
equivalent to the given procedure across histories. But, our invariant
allows function symbols. Use of the function symbol allows an OP
procedure to mimic different functions across histories. For instance,
procedure `remember' in Listing~\ref{lst:remember} mimics a constant
function for $\inv := g=0 \wedge init=0 \vee g= \F() \wedge
init=1$. Whereas, for the invariant $\inv := g=0 \wedge init=0 \vee
g=2 \wedge init=1$; the procedure mimics the function $\lambda n.2$.

% \section{Converting the program into logical formulae} \label{sec:approaches}

\subsection{Path condition generation}\label{sec:vcgen}
For our analysis we represent the given procedure in logic, such that
the representation captures the argument value at the beginning of the
procedure, the return value at the end of the procedure and the values
of the global variables at all the boundaries of the procedure. We
track the values of the global variables using extra variables as
explained in Section~\ref{sec:intermediate}. Thus, the set of free
variables in a path condition of a procedure are $X : \mi{\{n, gin,
  gout, g, \gbef_1 \cdots \gbef_m, \gaft_1 \cdots \gaft_m,}$ \\ ${\retVar, \F
  \}}$ all the remaining variables are universally quantified.

\begin{figure}
  \begin{align*}
    \pathCondition :=
    &(n \leq 1 \wedge retVar = 1 \wedge gout = g) \vee \\
    &(n > 1 \wedge g = -1 = gbef \wedge n = 19 \wedge temp1 = \F(18) \\
    \;&\wedge gaft1 = g1 \wedge g2 = 19 * temp1
    \wedge retVar = g2) \vee\\
    &(n > 1 \wedge \neg( g = -1 \wedge n = 19) \wedge g \neq -1
    \wedge n = 19 \wedge retVar = g = gout) \vee\\
    &(n > 1 \wedge n \neq 19 \wedge gbef2 = g \wedge temp2 = \F( n
    - 1) \wedge gaft2 = g1\\
    &\wedge retVar = n * temp2 \wedge gout = g1)\\
  \end{align*}
  \caption{Formula representing procedure `transformed factorial' in
    Listing~\ref{lst:factorialTransformed} (assuming that function
    $\F$ is equivalent to procedure factorial).}
  \label{fig:pathCondition}
\end{figure}

For example, the procedure `transformed factorial' in
Listing~\ref{lst:factorialTransformed} is expressed in logic as shown
in Figure~\ref{fig:pathCondition}. Each disjunct in
Figure~\ref{fig:pathCondition} represents a straight-line execution of
procedure factorial (from beginning, until end) . For instance, $(n <=
1 \wedge retVar = 1 \wedge gout = g)$ represents the case where $`n
\leq 1'$.

Representation of a program in logic is straight forward once it is
converted to our intermediate representation in
Section~\ref{sec:intermediate}. All the standard imperative statements
become conjuncts in the formula. `havoc' statements are omitted.
`assume x' are replaced with a conjunct `x' in the formula.

\subsection{Existential approach}\label{sec:existential}

\begin{figure}[htp]
  \begin{algorithm}[H]
    \begin{align*}
      \OPCheckE
       & \mi{(\inv :
        invariant, \pathCondition : path \; condition)} \equiv \\
      &\forall \{X - \{\F\}\}.\\
      &\inv \wedge \inv[\gaft_1/\g] \wedge \inv[\gaft_2/\g] \wedge
      \cdots \inv[\gaft_m/\g] \\
      &\wedge \pathCondition \implies ( \retVar = \F(\n)\\
            & \wedge \inv[\gout/\g] \wedge  \inv[\gbef_1/\g] \wedge
      \inv[\gbef_2/\g] \wedge \cdots inv[\gbef_m/\g]) \\
    \end{align*}
    \caption{Existential check : produces a formula whose
      satisfiability implies the given procedure is observationally
      pure} 
    \label{algo:someOPcheckCombined}
  \end{algorithm}  
\end{figure}

The existential approach in Algorithm~\ref{algo:someOPcheckCombined}
encodes a formula $\formula_e$. In formula $\formula_e$, the invariant
is assumed at the beginning of the procedure, and after each procedure
call ( $m$ procedure calls in total).  We abstract all the recursive
procedure calls in the original procedure using function symbol $\F$
in $\pathCondition$.  Next, the return value is constrained to be
equivalent to $\F(n)$, where `n' is the parameter value. Formula
$\formula_e$ has a single free variable $\F$, and it is constrained to
respect the given procedure, for all values of the parameter `n' and
all values of the global variable `g'. Thus, if $\formula_e$ is SAT,
it implies that there exists a function which mimics the given
procedure. 

The check encoded in Figure~\ref{algo:someOPcheckCombined} cannot be
expressed using standard requires and ensures statements (to the best
of our knowledge). In the standard pre-post verification, the
unreachability of the compliment of the post-condition is ensured. In
contrast, our analysis just requires that the post condition is
reachable. We check if $\retVar = \F(n)$ is satisfiable, but this
suffices for our analysis.

\subsection{Impurity witness approach}\label{sec:impurityWitness}

\begin{figure}[htp]
  \begin{algorithm}[H]
    \begin{align*}
      \OPCheckA &\mi{(\inv : invariant,
        \pathCondition: path\; condition)} \equiv &\\
      &\inv[\g_\alpha/\g] \wedge \inv[\g_\beta/\g]  & (1)\\
      & \wedge \inv[\gaft_{\alpha1}/\g] \wedge
      \inv[\gaft_{\alpha2}/\g] \cdots \wedge \inv[\gaft_{\alpha m}/\g]
      & (2)\\
      & \wedge \inv[\gaft_{\beta1}/\g] \wedge \inv[\gaft_{\beta2}/\g]
      \cdots \wedge \inv[\gaft_{\beta m}/\g] & (3)\\
      &\wedge \n_\alpha = \n_\beta  & (4)\\
      &\wedge \pathCondition[\g_\alpha/\g, \n_\alpha/\n, \retVar_\alpha/\retVar,
        \gout_\alpha/\gout, & (5)\\
        &\gbef_{\alpha1}/\gbef_1 \cdots, \gaft_{\alpha 1}/\gaft_1, \cdots
        \gaft_{\alpha m}/\gaft_m]  & (6)\\
      &\wedge \pathCondition[\g_\beta/\g, \n_\beta/\n, \retVar_\beta/\retVar,
        \gout_\beta/\gout, & (7)\\
        &\gbef_{\beta1}/\gbef_1 \cdots, \gaft_{\beta 1}/\gaft_1, \cdots
        \gaft_{\beta m}/\gaft_m] & (8)\\
      &\wedge (\retVar_\alpha \neq \retVar_\beta \vee \neg\inv[\gbef_{\alpha1}/\g] \vee
       \cdots \neg\inv[\gbef_{\alpha m}/\g] & (9) \\
        &\vee \neg\inv[\gout_\alpha/\g])  & (10)\\
    \end{align*}
    \caption{Impurity witness : produces a formula whose unsatisfiability
    implies observational purity.}
    \label{algo:pairwiseOPcheckCombined}
  \end{algorithm}  
\end{figure}

The impurity witness approach in
Algorithm~\ref{algo:pairwiseOPcheckCombined} encodes formula
$\formula_{iw}$. If $\formula_{iw}$ is UNSAT the given procedure is
OP. The idea encoded in $\formula_{iw}$ is that if a pair of traces of
the given procedure exists such that the traces start with the same
parameter value, and potentially different values for the global
variables return different results, then the procedure is not OP. In
this case, we have a witness to the non-OPness of the given
procedure. Otherwise, we prove that the given procedure is OP (this
proof will be mentioned in the final version of this paper).

The check encoded in Figure~\ref{algo:pairwiseOPcheckCombined}, cannot
be expressed in standard requires and ensures (again, to the best of
our knowledge). As the approach compares the given procedure with
itself, standard requires and ensures do not suffice. But works such as
\cite{lahiri2013differential} can be used to express this property. 

\subsection{Comparing the two approaches}

\begin{lstlisting}[caption={Procedure `bar': illustrates that
      existential approach is more precise that the impurity witness
      approach.}, label=lst:comparison]
int bar(int n) { 
  return g; 
}
\end{lstlisting}

The existential approach in Section~\ref{sec:existential} is more
precise than the impurity witness approach in
Section~\ref{sec:impurityWitness}. The existential approach requires
one mathematical function, such that it mimics the given procedure and
the invariant holds for this function to mark a procedure OP. On the
other hand, the impurity witness marks a procedure OP if for all
functions, the two instances of the path condition agree on the return
value. For instance procedure `bar' in Listing~\ref{lst:comparison} is
observationally pure for the invariant $(g = \F(0) \vee g = \F(1))$
and it mimics the mathematical function $\forall n. \F(0) = \F(1) =
\F(n)$ (thus, it must be a constant function). Now, the existential
approach marks this procedure as OP whereas the impurity witness
approach marks it as non-OP. The existential approach marks this
example OP, considering a function $\lambda n. 2$. Whereas there are
many functions $\F'$ such that $\F'(0) \neq \F'(1)$, now the procedure
`bar' is not OP for the invariant $(g = \F(0) \vee g =
\F(1))[\F'/\F]$.

Although the existential approach is more precise, the impurity
witness approach performs better with SAT-SMT solvers. In case of the
existential approach the solver has to find a satisfying assignment
for the free variable $\F$. Whereas, in case of the impurity witness
approach, the inequivalence is contradicted (if procedure is OP).
\section{Evaluation}\label{sec:experiments}

We have implemented our OP checking approach as a prototype using the Boogie
framework~\cite{leino2008boogie}, and have evaluated the approach using
this implementation on several examples. The objective of this evaluation
was primarily a sanity check, to test how our approach does on a set of
OP as well as non-OP procedures.

We tried several simple non-OP programs, and our implementation terminated
with a ``no'' answer on all of them.  We also tried the approach on several
OP procedures: (1) the `factCache' running example, (2) a version of
a factorial procedure that caches all arguments seen so far and their
corresponding return values in an array, (3) a version of factorial that caches
only the return value for argument value 19 in a scalar variable, (4) a
recursive procedure that returns the $n^\mathit{th}$ Fibonacci number and
caches all its arguments and corresponding return values seen so far in an
array, and (5) a ``matrix chain multiplication'' (MCM) procedure.
The last example  is based
on dynamic programming, and hence naturally uses a table to memoize 
results for sub-problems. Here, observational purity implies that the procedure always
returns the same solution for a given sub-problem, whether a hit was found
in the table or not.  The appendix of this paper depicts all the procedures
mentioned above as created by us directly in Boogie's language, as well as
the invariants that we supplied manually (in  SMT2
format).


It is notable that our ``existential approach'' causes the theorem prover
to not scale to even simple examples. The ``impurity witness'' approach
terminated on all the examples mentioned above with a ``yes'' answer,
with the theorem prover
taking less than 1 second on each example.
% This said, it is likely that
% more optimizations and heuristics would need to be developed on top of our
% prototype implementation to make it scale to large procedures.


%% \section{Conclusion}
\section{Related work}\label{sec:related}
Java modelling Language(JML)~\cite{leavens2008jml} is a specification
lanugage for Java Programs which uses Hoare style pre and post
conditions. Various specification languages such as JML and spec\#
allow provably pure procedure to be used in specification.
This overly restrictive constraint of using only provable pure
procedure was first overcome by Barnett~\cite{barnett200499}, and we
also sovle the same problem as them. However, we allow the
specification of a procedure to reference the same procedure.They also
note in the paper~\cite{barnett200499} that ``If we allow some P_g to
involve \textit{f}, then there is a circularity - it would take a
delicate argument, and additional conditions, to avoid unsoundness in
this case''. In this work, other than allowing specifications of a
procedure to self-reference we have also reduced the need for
annotations. Moreover, the annotations that are needed, we claim they
can be generated automatically for a large number of programs.

The idea of comparing a program to
itself~\cite{lahiri2013differential}~\cite{partush2013abstract} is
closely related to the impurity witness
approach. DAC~\cite{lahiri2013differential} uses this idea for finding
the assertions that failed in a version of a program with respect to
another version. Whereas, Partush and Yahav~\cite{partush2013abstract}
focus on finding the parameter values for which the two versions of
the procedure differ.

Naumann~\cite{naumann2007observational} calls a procedure OP if it is
output-equivalent to a procedure which is a side-effect free. And
side-effect-freeness does not entail OPness as the procedure's return
value may be dependent on some global variable. This work
presents/uses a theory for simulations, closely related to ownership
types. Thus this work is closely related but they solve a different
problem.

Cok~\cite{cok2008extensions} builds upon Barnett's
work~\cite{barnett2006allowing} and~\cite{barnett200499} suggests
partitioning the set of methods into ``pure'', ``secret'' and
``query'' methods, each scoped to a particular data group. The query
methods are OP. They give a set of rules to be followed by methods in
each group. Thus, structuring for modular reasoning about OPness.

Finifter~\cite{cok2008extensions} takes a different approach of
restricting the programming language in order to make OP checking
simpler. In the restricted language called Joe-E, OP methods have all
objects reachable from the parameters marked as immutable.

Salcianu~\cite{sualcianu2005purity} gives a static analysis that
checks if a method modifies the pre-existing state. All the operators
used in specifications do not have side effects. If a procedure does
not modify the pre-existing state it is called pure and
~\cite{barnett200499} suggests that such procedures can be used in
specifications. Our work in comparison generalizes further by allowing
procedures that modify pre-existing state but behave as mathematical
functions in specifications. We would like to point out that in their
work they manually marked library methods that did caching but were
semantically preserving as pure.

\nocite{barnett2004spec}
\nocite{lahiri2013differential}
\nocite{de2008z3}
\nocite{alpern1988detecting}
\nocite{sondergaard1990referential}
\nocite{flanagan2001avoiding}
\nocite{sualcianu2005purity}
\nocite{cytron1991efficiently}
\nocite{leino2008boogie}

%

%
% ---- Bibliography ----
%
%% \bibliographystyle{plainnat}
\bibliographystyle{splncs}
\bibliography{references}
%% \begin{thebibliography}{5}

%% \bibitem {clar:eke}
%% Clarke, F., a, I.:
%% Nonlinear oscillations and
%% boundary-value problems for Hamiltonian systems.
%% Arch. Rat. a. Anal. 78, 315--333 (1982)


%% \end{thebibliography}

%\appendix
%\begin{subappendices}
%\renewcommand{\thesection}{\Alph{section}}
%\section{Proofs}

\begin{lemma}
If there exists no impurity witness $(\pi_1,\pi_2)$ and there exists
no $\inv$-violation witness $\pi$, then $\proc$ satisfies $\pureinv$.
\end{lemma}

\begin{proof}
Consider the set $\Pi$ of all feasible executions of $\proc$.
If there exists no impurity witness, then there exists a function $f$ compatible with $\Pi$.
(We take the partial function $\{ (\inputval{\pi}, \outputval{\pi}) \; | \; \pi \in \Pi \}$ and
extend it to be a total function.)
Further, since there exists no $\inv$-violation witness, we are guaranteed that $(\pi,f) \models \inv$
for every $\pi \in \Pi$.
\end{proof}

\begin{lemma}
If there exists a $\inv$-violation witness $\pi$, then there exists a minimal $\inv$-violation witness.
\end{lemma}

\begin{lemma}
If there exists an impurity witness, then there exists a minimal impurity witness or a minimal
$\inv$-violation witness.
\end{lemma}
\section{Examples}

\subsection{Factorial}
\begin{verbatim}
invariant : (forall ((k Int) ) (or (= (select g k) 0) (= (*
k (my_FactArray (- k 1))) (select g k))))
\end{verbatim}
\begin{lstlisting}[language=c, caption= {Procedure `factorial' :
      returns factorial of `n' and memoizes result for argument value
      `19'.}, label=lst:factImpl]
var g: [int] int;
procedure {:entrypoint} FactArray(n: int) returns (r: int) modifies g;{
  var k :int;
  if( n <= 1) { r := 1;}
  else {
    if( g[n] == 0) {
      call k := FactArray(n - 1);
      g[n] := k * n;
    } 
    r  := g[n];
  }
}
\end{lstlisting}

\subsection{Factorial Recent}
\begin{verbatim}
Invariant : (or (= lastAns (- 0 1)) (= lastAns (* (recent_fact (+ (- 0 1) lastN)) lastN)))
\end{verbatim}
%% \lstinputlisting{examples/fact-recent/factRecent.bpl}
\begin{lstlisting}[language=c, caption= {Procedure `recent\_fact' :
      returns factorial of `n' and memoizes result for the last argument value.}, label=lst:factorialRecent]
var lastN: int;
var lastAns: int;
//invariant : lastAns = -1 || g = lastN * factorial(lastN -1) && lastN >1 
procedure {:entrypoint} recent_fact(a: int) returns (r: int) modifies lastN, lastAns;{
  if( a <= 1) { r := 1;}
  else {
     if( a == lastN && lastAns != -1) {
        r := lastAns;
      } else {
      call r := recent_fact( a - 1);    
      r := a * r;
      lastN := a;
      lastAns := r;
    }
  }
}
\end{lstlisting}

\subsection{Fibonacci}
%% \lstinputlisting{examples/fib/fib.bpl}
\begin{verbatim}
Invariant : (or (= (select cache k) 0) (= (+ (fib (- k 1)) (fib (- k 2))) (select cache k)))
\end{verbatim}
\begin{lstlisting}[language=c, caption= {Procedure `fib' :
      returns the n'th fibonacci number}, label=lst:factorialSimple]
//invariant : forall k. cache[k] = 0 OR cache[k] = fib(k -1) + fib( k -2)
var cache:[int] int;
procedure {:entrypoint} fib(n: int) returns (r: int) modifies cache;{
  var a, b : int;
  if( n <= 2) {
    r := 1;
  } else {
    if(cache[n] != 0) {
      r := cache[n];
    } else {
      call a := fib(n -1);
      call b := fib(n -2);
      r := a + b;
      cache[n] := r;
    }
  }
}
\end{lstlisting}

\subsection{Matrix Chain Multipication}
\begin{verbatim}
(or (= (select m i j) (- 0 1)) (= (my_foo i j i (- 0 1)) (select m i
 j)))
\end{verbatim}

\begin{lstlisting}[language=c, caption= {Procedure `matrix chain multipication' :
      returns the minimum number of multipications needed to multiply a sequence of matrices.}, label=lst:factorialRecent]
var p: [int] int;
var m: [int, int] int;
//invariant : \forall {i, j}. m[i, j] = -1 OR m[i, j] = foo(i, j, i, infinity)
procedure {:entrypoint} mcm(i: int, j: int) returns (r: int) modifies m;{
        var k, q : int;
        var a, b :int;
        if(i == j) {
                m[i, j] := 0;
                r := 0;
        } else {
                if( m[i, j] > 0) {
                        r := m[i, j];
                } else {
                        k := i;
                        call r := foo(i, j, k, m[i, j]);
                        m[i, j] := r;
                }
        }
}

procedure foo(i: int, j: int, k:int, min :int) returns (r: int) modifies m;{
        var a, b, q : int;
        var min1 :int;
        if(k >= j) {
                r:= min;
        } else {        
                call a := mcm(i, k);
                call b := mcm(k+1, j);
                q := a + b + p[i-1] * p[k] * p[j];
                if( q < min) {
                  min1 := q;
                } else {
                  min1 := min;
                }
                call r := foo(i, j, k + 1, min1);
        }       
}
\end{lstlisting}

%\end{subappendices}

\end{document}
