\subsection{Our intermediate representation}
\label{sec:intermediate}
We transform the given procedure into the following representation
in-order to encode it in logic. For example procedure factorial
in Listing~\ref{lst:factorialTransformed}, is the transformed version
of procedure factorial in Listing~\ref{lst:factorialSimple}.

Our analysis expects the following :
\begin{enumerate}
\item Procedure calls are approximated using function symbols. The
  statement `x = foo(y)' is replaced with `x = $\F$(y)'. Since
  procedure calls may modify global variables, we add the statement
  `havoc g' for each global variable `g' accessible from the
  procedure.
\item Next, the procedure must have extra variables to store the value
  of global variables at procedure boundaries. Before the
  $\mathit{i^{th}}$ procedure call, we add the statement `gbef$_i$ =
  g' and after it we add `gaft$_i$ = g'. Similarly, we add the
  statement `gout = g' after the assignment to the variable `\retVar'.
\item The procedure should be in static single assignment (SSA)
  form. The procedure should be converted to SSA after the above
  mentioned points have been satisfied.
\end{enumerate}

\begin{lstlisting}[language=c, caption= {Procedure factorial from
      Listing~\ref{lst:factorialSimple} converted to the form our
      approach expects. We refer to this procedure as `transformed
      factorial'.}, label=lst:factorialTransformed]
int g = -1;
int transformedFactorial( int n) { // redo
  if(n <= 1) {
    retVar = 1;
    gout = g;
  } else if(g == -1 && n == 19) {
    gbef1 = g;
    temp1 = F(18);  // temp1 = factorial(18)
    havoc(g1);
    gaft1 = g1;
    g2 = 19 * temp1;
    retVar = g2;
    gout = g2;
  } else if (g != -1 && n == 19) {
    retVar = g;
    gout = g;
  } else {
    gbef2 = g;
    temp2 = F( n - 1 );  //temp2 = factorial(n-1)
    havoc(g2);
    gaft2 = g3;
    retVar = n * temp2;
    gout = g3;
  }
}
\end{lstlisting}

%% In procedure `transformedFactorial'
%% Listing~\ref{lst:factorialTransformed}, in comparison to procedure
%% factorial in Listing~\ref{lst:factorialSimple}, the return statement
%% (line 4) is replaced by an assignment to variable `\retVar' (line
%% 4). After line 5 of `transformed factorial', an extra variable `gout'
%% is assigned the value of global variable `g' (value of `g' at end of
%% program).  Similarly, variable `gbef1' (line 7, `transformed
%% factorial') is added to capture the value of the global variable
%% before the procedure call (a program boundary), and variable `gaft1'
%% is inserted at line 11 to capture the value of `g' after the procedure
%% call. Also, havoc statements at line 8, 20 and 9, 21 over-approximate
%% the return from the procedure call statement and updates to the global
%% variable respectively. The procedure call statement is substituted
%% with function symbols in line 10 and 22, accounting the given
%% procedure as a function.

Now we compare `transformed factorial' and procedure factorial from
Listings~\ref{lst:factorialTransformed} and \ref{lst:factorialSimple}
respectively. The return statement (line 4) in procedure factorial
is replaced by assignment to variable `\retVar' (line 4). Also, we
have a added an assignment statement (line 5) in `transformed
factorial', that defines variable `gout' in-order to store the value
of variable `g' at the procedure boundary. Similarly, variable `gbef1'
(line 7) is added in `transformed factorial' to capture the value of
the global variable before the procedure call and assignment to
variable `gaft1' is inserted at line 10 to capture the value of `g'
after the procedure call. Also, havoc statements at lines 9 and 20
over-approximate any side effects to the global variables. And in
lines 8 and 19, the procedure call statements are substituted for
function symbols.

\subsection{Converting the program into logical formulae} \label{sec:approaches}

\subsection{Path condition generation}\label{sec:vcgen}
For our analysis we represent the given procedure in logic, such that
the representation captures the argument value at the beginning of the
procedure, the return value at the end of the procedure and the values
of the global variables at all the boundaries of the procedure. We
track the values of the global variables using extra variables as
explained in Section~\ref{sec:intermediate}. Thus, the set of free
variables in a path condition of a procedure are $X : \mi{\{n, gin,
  gout, g, \gbef_1 \cdots \gbef_m, \gaft_1 \cdots \gaft_m,}$ \\ ${\retVar, \F
  \}}$ all the remaining variables are universally quantified.

\begin{figure}
  \begin{align*}
    \pathCondition :=
    &(n \leq 1 \wedge retVar = 1 \wedge gout = g) \vee \\
    &(n > 1 \wedge g = -1 = gbef \wedge n = 19 \wedge temp1 = \F(18) \\
    \;&\wedge gaft1 = g1 \wedge g2 = 19 * temp1
    \wedge retVar = g2) \vee\\
    &(n > 1 \wedge \neg( g = -1 \wedge n = 19) \wedge g \neq -1
    \wedge n = 19 \wedge retVar = g = gout) \vee\\
    &(n > 1 \wedge n \neq 19 \wedge gbef2 = g \wedge temp2 = \F( n
    - 1) \wedge gaft2 = g1\\
    &\wedge retVar = n * temp2 \wedge gout = g1)\\
  \end{align*}
  \caption{Formula representing procedure `transformed factorial' in
    Listing~\ref{lst:factorialTransformed} (assuming that function
    $\F$ is equivalent to procedure factorial).}
  \label{fig:pathCondition}
\end{figure}

For example, the procedure `transformed factorial' in
Listing~\ref{lst:factorialTransformed} is expressed in logic as shown
in Figure~\ref{fig:pathCondition}. Each disjunct in
Figure~\ref{fig:pathCondition} represents a straight-line execution of
procedure factorial (from beginning, until end) . For instance, $(n <=
1 \wedge retVar = 1 \wedge gout = g)$ represents the case where $`n
\leq 1'$.

Representation of a program in logic is straight forward once it is
converted to our intermediate representation in
Section~\ref{sec:intermediate}. All the standard imperative statements
become conjuncts in the formula. `havoc' statements are omitted.
`assume x' are replaced with a conjunct `x' in the formula.