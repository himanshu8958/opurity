\section{Examples}

\subsection{Factorial}
source : https://gist.github.com/gnke/1144837
this is a line by line translation.
%% \lstinputlisting{examples/factorial/fact1.bpl}

\begin{lstlisting}[language=c, caption= {Procedure `factorial' :
      returns factorial of `n' and memoizes result for argument value
      `19'.}, label=lst:factImpl]
var nineteen: int;
procedure {:entrypoint} Fact(a: int) returns (r: int) modifies nineteen;{
  if( a <= 1) { r := 1;}
  else {
    if( a == 19) {
      if( nineteen == -1) {
        call nineteen := Fact(18);
        nineteen := nineteen * 19;
        r := nineteen;
      } else {
        r := nineteen;
      }
    } else {
      call r := Fact( a - 1);
      r := a * r;
    }
  }
}
\end{lstlisting}

\subsection{Factorial Recent}
%% \lstinputlisting{examples/fact-recent/factRecent.bpl}
\begin{lstlisting}[language=c, caption= {Procedure `recent\_fact' :
      returns factorial of `n' and memoizes result for the last argument value.}, label=lst:factorialRecent]
var lastN: int;
var lastAns: int;
//invariant : lastAns = -1 || g = lastN * factorial(lastN -1) && lastN >1 
procedure {:entrypoint} recent_fact(a: int) returns (r: int) modifies lastN, lastAns;{
  if( a <= 1) { r := 1;}
  else {
     if( a == lastN && lastAns != -1) {
        r := lastAns;
      } else {
      call r := recent_fact( a - 1);    
      r := a * r;
      lastN := a;
      lastAns := r;
    }
  }
}
\end{lstlisting}

\subsection{Fibonacci}
%% \lstinputlisting{examples/fib/fib.bpl}
\begin{lstlisting}[language=c, caption= {Procedure `fib' :
      returns the n'th fibonacci number}, label=lst:factorialSimple]
//invariant : forall k. cache[k] = 0 OR cache[k] = fib(k -1) + fib( k -2)
var cache:[int] int;
procedure {:entrypoint} fib(n: int) returns (r: int) modifies cache;{
  var a, b : int;
  if( n <= 2) {
    r := 1;
  } else {
    if(cache[n] != 0) {
      r := cache[n];
    } else {
      call a := fib(n -1);
      call b := fib(n -2);
      r := a + b;
      cache[n] := r;
    }
  }
}
\end{lstlisting}
