\section{Examples}

\subsection{Factorial, caching single result}
Invariant:
\begin{verbatim}
(or (= nineteen (- 0 1)) (= nineteen (* (FactSingle 18) 19)))
\end{verbatim}

\begin{lstlisting}[language=c, caption= {Returns factorial of `n',
      and memoizes result for argument value
      `19' in global variable `nineteen'.}, label=lst:fact19]
var nineteen: int;
/* invariant: nineteen = -1 ||
  nineteen = 19*FactSingle(18)   */
procedure {:entrypoint} FactSingle(n: int) returns (r: int) modifies nineteen;{
  if( n <= 1) { r := 1;}
  else {
    if( n == 19) {
      if( nineteen == -1) {
        call nineteen := FactSingle(18);
        nineteen := nineteen * 19;
        r := nineteen;
      } else {
        r := nineteen;
      }
    } else {
      call r := FactSingle( n - 1);
      r := n * r;
    }
  }
}
\end{lstlisting}

\subsection{Factorial, caching in  array}
Invariant:
\begin{verbatim}
(forall ((k Int) ) (or (= (select g k) 0) (= (*
k (FactArray (- k 1))) (select g k))))
\end{verbatim}
\begin{lstlisting}[language=c, caption= {Returns factorial of `n', 
      and caches results for all argument values greater than 1 in global
    array `g'},
    label=lst:factArrayImpl]
var g: [int] int;
/* invariant: forall k. g[k] = 0 ||
  g[k] = k*FactArray(k-1)   */  
procedure {:entrypoint} FactArray(n: int) returns (r: int) modifies g;{
  var k :int;
  if( n <= 1) { r := 1;}
  else {
    if( g[n] == 0) {
      call k := FactArray(n - 1);
      g[n] := k * n;
    } 
    r  := g[n];
  }
}
\end{lstlisting}

\subsection{Factorial, caching only last-seen argument}
Invariant:
\begin{verbatim}
(or (= g (- 0 1)) (= g (* (FactRecent (+ (- 0 1) lastN)) lastN)))
\end{verbatim}
%% \lstinputlisting{examples/fact-recent/factRecent.bpl}
\begin{lstlisting}[language=c, caption= {Returns factorial of `n', 
      and caches last seen argument in lastN and corresponding return value
    in g}, label=lst:factorialRecent]
var lastN: int;
var g: int;
//invariant : g = -1 || g = lastN * FactRecent(lastN - 1) 
procedure {:entrypoint} FactRecent(n: int) returns (r: int) modifies lastN, g;{
  if( n <= 1) { r := 1;}
  else {
     if( n == lastN && g != -1) {
        r := g;
      } else {
      call r := FactRecent( n - 1);    
      r := n * r;
      lastN := a;
      g := r;
    }
  }
}
\end{lstlisting}

\subsection{Fibonacci, caching in array}
%% \lstinputlisting{examples/fib/fib.bpl}
Invariant:
\begin{verbatim}
(forall (k Int) (or (= (select cache k) 0) (= (+ (fib (- k 1)) (fib (- k 2)))
  (select cache k))))
\end{verbatim}
\begin{lstlisting}[language=c, caption= {Returns $n$th Fibonacci number, and
      caches all arguments and return values seen so far in global array `cache'}]
/* invariant : forall k. cache[k] = 0 OR cache[k] = fib(k -1) + fib( k -2) */
var cache:[int] int;
procedure {:entrypoint} fib(n: int) returns (r: int) modifies cache;{
  var a, b : int;
  if( n <= 2) {
    r := 1;
  } else {
    if(cache[n] != 0) {
      r := cache[n];
    } else {
      call a := fib(n -1);
      call b := fib(n -2);
      r := a + b;
      cache[n] := r;
    }
  }
}
\end{lstlisting}

\subsection{Matrix Chain Multiplication}
Invariant:
\begin{verbatim}
 (forall ((i Int) (j Int)) (or (= (select m i j) (- 0 1))
   (= (chooseSplit i j i (- 0 1)) (select m i j))))
\end{verbatim}

\begin{lstlisting}[language=c, caption= {Returns the minimum number of multipications needed to multiply a sequence of matrices.}, label=lst:mcm]
var p: [int] int;
var m: [int, int] int;
/* invariant : forall i, j. m[i, j] = -1 OR m[i, j] = chooseSplit(i, j, i, -1) */
procedure {:entrypoint} mcm(i: int, j: int) returns (r: int) modifies m;{
        var k, q : int;
        var a, b :int;
        if(i == j) {
                m[i, j] := 0;
                r := 0;
        } else {
                if( m[i, j] > 0) {
                        r := m[i, j];
                } else {
                        k := i;
                        call r := chooseSplit(i, j, k, m[i, j]);
                        m[i, j] := r;
                }
        }
}

procedure chooseSplit(i: int, j: int, k:int, min :int) returns (r: int) modifies m;{
        var a, b, q : int;
        var min1 :int;
        if(k >= j) {
                r:= min;
        } else {        
                call a := mcm(i, k);
                call b := mcm(k+1, j);
                q := a + b + p[i-1] * p[k] * p[j];
                if( q < min) {
                  min1 := q;
                } else {
                  min1 := min;
                }
                call r := chooseSplit(i, j, k + 1, min1);
        }       
}
\end{lstlisting}

\subsubsection{Notes about how this program was verified.}
In this example, `p' is an input array, and stores the dimensions of the matrices
to be multiplied. p$[i]$ and p$[i+1]$ store the dimensions of the $i^\mathit{th}$
matrix to be multiplied. Table `m' is computed by the procedure. m$[i,j]$
stores the minimum number of scalar multiplications required
to multiply the sequence of matrices from the $i^\mathit{th}$ matrix
through the $j^\mathit{th}$ matrix. The procedure `mcm'
basically returns m[i,j] given arguments i and j. chooseSplit is a
tail-recursive procedure, which is used to consider all possible k, i $<$ k
$\leq$ j, at which to split the sequence $i, i+1, \ldots, j$. This search
is required to find the optimal split point.

Note, mcm and chooseSplit call each other. Although we presented our
approach as if it can check a given single procedure, it can also check if
a set of mutually-recursive procedures are all OP. The idea is that each
procedure needs an invariant, and this invariant can use function symbols
to refer to all procedures called by this procedure. Each procedure can
then be checked independently using our approach as-is. Intuitively, this
works because we check all the procedures, and while checking any procedure
we assume that the other procedures are OP.

The invariant shown in the example above is for procedure `mcm'. We applied
our tool only on procedure mcm. We did not apply our tool on procedure
`chooseSplit', because currently our tool is not set up analyze all
procedures in a multi-procedure program. However, it is obvious that the
impurity witness approach would certify procedure chooseSplit as OP, even
without any invariant, because this procedure does not refer to or update
the mutable global variable `m'. Therefore, for any tuple of argument
values, the procedure would always follow the same path no matter how many
times this tuple is repeated. Now, if the calls encountered in this path
are assumed to be to observationally pure procedures (inductive
hypothesis), then the path must return the same value every time.
