\section{Evaluation}\label{sec:experiments}

We have implemented our OP checking approach as a prototype using the Boogie
framework~\cite{leino2008boogie}, and have evaluated the approach using
this implementation on several examples. The objective of this evaluation
was primarily a sanity check, to test how our approach does on a set of
OP as well as non-OP procedures.

We tried several simple non-OP programs, and our implementation terminated
with a ``no'' answer on all of them.  We also tried the approach on several
OP procedures: (1) the `factCache' running example, (2) a version of
a factorial procedure that caches all arguments seen so far and their
corresponding return values in an array, (3) a version of factorial that caches
only the return value for argument value 19 in a scalar variable, (4) a
recursive procedure that returns the $n^\mathit{th}$ Fibonacci number and
caches all its arguments and corresponding return values seen so far in an
array, and (5) a ``matrix chain multiplication'' (MCM) procedure.
The last example  is based
on dynamic programming, and hence naturally uses a table to memoize 
results for sub-problems. Here, observational purity implies that the procedure always
returns the same solution for a given sub-problem, whether a hit was found
in the table or not.  The appendix of a technical report associated with
this paper  depicts all the procedures
mentioned above as created by us directly in Boogie's language, as well as
the invariants that we supplied manually (in  SMT2
format).


It is notable that our ``existential approach'' causes the theorem prover
to not scale to even simple examples. The ``impurity witness'' approach
terminated on all the examples mentioned above with a ``yes'' answer,
with the theorem prover
taking less than 1 second on each example.
% This said, it is likely that
% more optimizations and heuristics would need to be developed on top of our
% prototype implementation to make it scale to large procedures.

